\subsection{Поставщики типов в языке F\#}\label{f-sharp-type-providers}
Одновременно с выпуском версии 3.0 языка F\# его разработчики сообщили о введении новой концепции развития~--- информационно насыщенном программировании (Information Rich Programming)~\cite{joepamer2011}.
Одним из нововведений, поддерживающим эту концепцию являются поставщики типов (type providers).
Поставщики типов позволяют получить статически типизированный доступ к различным источникам данных.
В стандартной поставке языка F\# существуют реализации поставщиков типов для работы с базами данных, с серверами, поддерживающими протокол OData (Open Data Protocol)\footnote{\url{http://www.odata.org/developers/protocols/}} и другие\footnote{\url{http://msdn.microsoft.com/en-us/library/hh156509(v=vs.110).aspx}.}.

\subsubsection{Примеры}
\begin{description}
\item[Поставщик типов для файлов с ресурсами.]
Рассмотрим файл с ресурсами\footnote{\url{http://msdn.microsoft.com/en-us/library/ekyft91f(v=vs.71).aspx}} \path{Resource.resx}, внутри которого определены следующие пары <<ключ-значение>>:
\begin{code}
\begin{lstlisting}[caption={Пример файла с ресурсами.}, label=resource-resx]
<data name="string1">
    <value>First</value>
</data>
<data name="string2">
    <value>Second</value>
</data>
\end{lstlisting}
\end{code}

Благодаря механизму загрузки типов можно получить статически типизированный доступ к значениям этих полей.

\clearpage

Рассмотрим следующий пример использования загрузчика типов для файла с ресурсами:

\begin{code}
\lstinputlisting[caption={Пример программы на языке F\#, которая использует механизм загрузки типов из файла с ресурсами, приведенного на листинге \ref{resource-resx}.}, label=resource-type-provider]{listings/resource_type_provider.fs}
\end{code}
\begin{code}
Результат выполнения программы:
\begin{lstlisting}[caption={Результат выполнения программы \ref{resource-type-provider}.}]
string1 from resource is First
string2 from resource is Second
\end{lstlisting}
\end{code}

Самое важное, что стоит отметить в этом примере~--- статический доступ к полям \path{T.Resource.String1} и \path{T.Resource.String2}. Если изменить ключи в файле с ресурсами, то программа не скомпилируется~--- случится ошибка времени компиляции, а не времени исполнения. Такой подход позволяет снизить число ошибок, тем самым снижает затраты на тестирование программ.

Небольшое пояснение~--- атрибут \texttt{[<Generate>]} указывает компилятору на то, что тип \texttt{T} будет сгенерирован, поэтому его не требуется выводить.

\clearpage

\item[Поставщик типов для файлов баз данных.] Таблица \path{students} содержится в файле \path{DataClasses.dbml}.
\begin{code}
\lstinputlisting[caption={Пример программы на языке F\#, использующей механизм загрузки типов из файла базы данных.}, label=dbml-type-provider]{listings/dbml_type_provider.fs}
\end{code}

Результат выполнения программы:
\begin{lstlisting}[caption={Результат выполнения программы \ref{dbml-type-provider}.}, label=dbml-result]
student name = Anita
student name = Ken
student name = Cathy
\end{lstlisting}

\end{description}

Аналогично приведенному выше примеру, важно отметить статически типизированный доступ к таблице базе данных~--- \path{db.Students} и столбцу строки~--- \path{n.Name}. Опять же, по аналогии с примером выше, при изменении схемы базы данных, случится ошибка времени компиляции, а не времени исполнения.

В приведенном примере используется язык запросов~--- LINQ~\cite{linq}.
Доступ к нему из языка F\# становится доступным благодаря специальному Query Expression\footnote{\url{http://msdn.microsoft.com/en-us/library/hh225374(v=vs.110).aspx}}.

Приведенные примеры позволяют взглянуть на существующие на сегодняшний день реализации механизма загрузки типов в языке F\#.
Поставщики типов позволяют получать статически типизированный доступ к различным источникам информации с известным форматом.

Реализации таких механизмов появились как часть концепции развития языка F\#.
На текущий момент, разработчики языка позиционируют свой продукт как основной инструмент для написания приложений для извлечения и обработки данных на платформе .NET~\cite{duffy2008}.
В рамках такого подхода, поставщики типов позволяют эффективно получать доступ к источникам информации разного рода.

Дополнительным плюсами такого подхода являются согласованность прикладного приложения и схемы источника данных и отсутствие ошибок времени исполнения, связанных с изменением схемы данных (вместо них будут происходить ошибки времени компиляции, которые намного проще обнаружить и исправить).