\subsection{Загрузчики типов в языке Gosu}\label{gosu-type-loaders}
В языке Gosu~\cite{gosuguide} существует открытая система типов (The Open Type System), которая позволяет реализовывать собственные загрузчики типов (type loaders) для любых источников.
Существенным минусом этой реализации является создание типов загружаемых данных на этапе выполнения программы, а не на этапе компиляции.
В стандартной библиотеке языка Gosu существуют загрузчики для получения типов данных из XSD, WSDL и т.д.

\subsubsection{Примеры}

Рассмотрим следующий пример проекта на языке Gosu, который показывает работу загрузчика типов из XML схем
со следующим расположением файлов:

\begin{lstlisting}[caption={Расположение файлов в примере загрузчиков типов на языке Gosu. \td}, label=gosu-project-layout]
/test_project 
  /src 
    /xsds
      employee.xsd 
    /bin 
      test_project.gsp
\end{lstlisting}

\todo{Рассмотрим} подробнее подключаемую схему:

\begin{lstlisting}[caption={Содержимое файла \texttt{employee.xsd} из примера~\ref{gosu-project-layout}.}, label=gosu-employee-xsd]
<xsd:schema xmlns:xsd="http://www.w3.org/2001/XMLSchema">
    <xsd:element name="Employee">
        <xsd:complexType>
            <xsd:sequence>
                <xsd:element name="SSN" type="xsd:string"/>
                <xsd:element name="Name" type="xsd:string"/>
                <xsd:element name="DateOfBirth" type="xsd:date"/>
                <xsd:element name="EmployeeType" type="xsd:string"/>
                <xsd:element name="Salary" type="xsd:long"/>
            </xsd:sequence>
        </xsd:complexType>
    </xsd:element>
</xsd:schema>
\end{lstlisting}

Можно заметить, что в ней описывается сущность \path{Employee} с пятью полями: строковыми 
\path{SSN}, \path{Name}, \path{EmployeeType}, целочисленным \path{Salary}
и одно поле с датой~--- \path{DateOfBirth}.

Подключение схемы в программу на языке Gosu происходит с помощью добавления директории с файлом схемы в \path{classpath}.
После этого, класс \path{Employee} загружается и становится доступным для использования.

На следующем примере можно видеть создание экземпляра этого класса и заполнение его полей необходимыми данными:

\begin{lstlisting}[caption={Содержимое файла \texttt{test\_project.gsp} из примера~\ref{gosu-project-layout}.}, label=gosu-xsd-loader]
classpath "../src"
uses xsds.employee.Employee
uses gw.xml.date.XmlDate
var x = new Employee() { 
    :SSN = "1011001", 
    :Name = "Ted", 
    :DateOfBirth = new XmlDate(), 
    :EmployeeType = "None", 
    :Salary=100000 
}
x.print()
\end{lstlisting}

\todo{Выведет на консоль: }



