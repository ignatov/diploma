\subsection{Загрузчики типов в языке Gosu}\label{gosu-type-loaders}
В языке Gosu~\cite{gosuguide} существует открытая система типов (Open Type System), которая позволяет реализовывать собственные загрузчики типов (type loaders) для любых источников.
Существенным минусом этой реализации является создание типов загружаемых данных на этапе выполнения программы, а не на этапе компиляции.
В стандартной библиотеке языка Gosu существуют загрузчики для получения типов данных из XSD, WSDL и т.д.

\subsubsection{Примеры}

\begin{description}
\item[Загрузчик типов из XML схем.] Рассмотрим следующий пример проекта на языке Gosu, который показывает работу загрузчика типов из XML схем расположением файлов, показанном в листинге \ref{gosu-project-layout}.

\begin{code}\begin{lstlisting}[caption={Расположение файлов в примере загрузчиков типов на языке Gosu, использующее загрузку типов из XML схем.}, label=gosu-project-layout]
/test_project
  /src
    /xsds
      employee.xsd
    /bin
      test_project.gsp
\end{lstlisting}\end{code}

\begin{code}
Рассмотрим подробнее подключаемую схему:
\begin{lstlisting}[caption={Содержимое файла \texttt{employee.xsd} из примера~\ref{gosu-project-layout}.}, label=gosu-employee-xsd]
<xsd:schema xmlns:xsd="http://www.w3.org/2001/XMLSchema">
    <xsd:element name="Employee">
        <xsd:complexType>
            <xsd:sequence>
                <xsd:element name="SSN" type="xsd:string"/>
                <xsd:element name="Name" type="xsd:string"/>
                <xsd:element name="DateOfBirth" type="xsd:date"/>
                <xsd:element name="EmployeeType" type="xsd:string"/>
                <xsd:element name="Salary" type="xsd:long"/>
            </xsd:sequence>
        </xsd:complexType>
    </xsd:element>
</xsd:schema>
\end{lstlisting}\end{code}

Можно заметить, что в ней описывается сущность \path{Employee} с пятью полями: строковыми
\path{SSN}, \path{Name}, \path{EmployeeType}, целочисленным \path{Salary}
и одно поле с датой~--- \path{DateOfBirth}.

Подключение схемы в программу на языке Gosu происходит с помощью добавления директории с файлом XML схемы в переменную окружения \path{classpath}.
После этого, класс \path{Employee} загружается и становится доступным для использования.

На следующем примере можно видеть создание экземпляра этого класса и заполнение его полей необходимыми данными:

\begin{code}\begin{lstlisting}[caption={Содержимое файла \texttt{test\_project.gsp} из примера~\ref{gosu-project-layout}.}, label=gosu-xsd-loader]
classpath "../src"
uses xsds.employee.Employee
uses gw.xml.date.XmlDate
var x = new Employee() {
    :SSN = "1011001",
    :Name = "Ted",
    :DateOfBirth = new XmlDate(),
    :EmployeeType = "None",
    :Salary=100000
}
x.print()
\end{lstlisting}

Выведет на консоль:
\begin{lstlisting}[language=xml]
<?xml version="1.0"?>
<Employee>
  <SSN>1011001</SSN>
  <Name>Ted</Name>
  <DateOfBirth>2012-02-09</DateOfBirth>
  <EmployeeType>None</EmployeeType>
  <Salary>100000</Salary>
</Employee>
\end{lstlisting}

\end{code}


\item[Загрузчик типов для создания объектно-реляционного отображения]. Другим примером может служить одна из частей каркаса для построения веб-приложений (web application framework), написанного на языке Gosu~--- Ronin\footnote{\url{http://ronin-web.org/}}.

А именно та часть, которая позволяет получать объектно-реляционное отображение через драйвер JDBC~\cite{jdbc-book} посредством загрузчиков типов, доступных в языке Gosu.

Таблицы отображаются в типы, а столбцы~--- в свойства; при этом отсутствует генерация кода для взаимодействия с базой данных и предоставления объектно-реляционного отображения с помощью автоматически созданных <<классов-оберток>>; также удается избежать многословной настройки соединения, привычной для программ на языке Java.
Вышеперечисленные плюсы позволяют создавать веб-приложения элегантным способом.

\end{description}