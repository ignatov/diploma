\section{Обзор средств расширения систем типов}
В разделе~\ref{f-sharp-type-providers} данной главы рассматривается механизм поставщиков типов языка F\#~3.0. Далее, в разделе~\ref{gosu-type-loaders} рассказывается про открытую систему типов и механизм загрузчиков типов языка Gosu. А в разделе~\ref{similar-mechanisms} приводится обзор сходных механизмов расширения типов языков программирования.

\subsection{Поставщики типов в языке F\#}\label{f-sharp-type-providers}
Одновременно с выпуском версии 3.0 языка F\# его разработчики сообщили о введении новой концепции развития~--- информационно насыщенном программировании (Information Rich Programming)~\cite{joepamer2011}.
Одним из нововведений, поддерживающим эту концепцию являются поставщики типов (type providers).
Поставщики типов позволяют получить статически типизированный доступ к различным источникам данных.
В стандартной поставке языка F\# существуют реализации поставщиков типов для работы с базами данных, с серверами, поддерживающими протокол OData (open data protocol) и другие\footnote{Информация о поставщиках типов доступна по адресу \url{http://msdn.microsoft.com/en-us/library/hh156509(v=vs.110).aspx}.}.

\subsubsection{Примеры}
\begin{description}
\item[Поставщик типов для файлов с ресурсами.] Строки \path{string1} и \path{string2} определены внутри файла \path{Resource.resx}.

\lstinputlisting{listings/resource_type_provider.fs}
Результат выполнения:
\begin{code}\begin{lstlisting}
string1 from resource is First
string2 from resource is Second
\end{lstlisting}\end{code}

\item[Поставщик типов для файлов баз данных.] Таблица \path{students} содержится в файле

\path{DataClasses.dbml}.

\lstinputlisting{listings/dbml_type_provider.fs}
Результат выполнения:
\begin{code}\begin{lstlisting}
student name = Anita
student name = Ken
student name = Cathy
\end{lstlisting}\end{code}

\end{description}
\subsection{Загрузчики типов в языке Gosu}\label{gosu-type-loaders}
В языке Gosu~\cite{gosuguide} существует открытая система типов (Open Type System), которая позволяет реализовывать собственные загрузчики типов (type loaders) для любых источников.
Существенным недостатком этой реализации является создание типов загружаемых данных на этапе выполнения программы, а не на этапе компиляции.
В стандартной библиотеке языка Gosu существуют загрузчики для получения типов данных из XSD, WSDL и т.д.

\subsubsection{Примеры}

\begin{description}
\item[Загрузчик типов из XML схем.] Рассмотрим следующий пример проекта на языке Gosu, который показывает работу загрузчика типов из XML схем расположением файлов, показанном в листинге \ref{gosu-project-layout}.

\begin{code}\begin{lstlisting}[caption={Расположение файлов в примере загрузчиков типов на языке Gosu, использующее загрузку типов из XML схем.}, label=gosu-project-layout]
/test_project
  /src
    /xsds
      employee.xsd
    /bin
      test_project.gsp
\end{lstlisting}\end{code}

\begin{code}
Рассмотрим подробнее подключаемую схему:
\begin{lstlisting}[caption={Содержимое файла \texttt{employee.xsd} из примера~\ref{gosu-project-layout}.}, label=gosu-employee-xsd]
<xsd:schema xmlns:xsd="http://www.w3.org/2001/XMLSchema">
    <xsd:element name="Employee">
        <xsd:complexType>
            <xsd:sequence>
                <xsd:element name="SSN" type="xsd:string"/>
                <xsd:element name="Name" type="xsd:string"/>
                <xsd:element name="DateOfBirth" type="xsd:date"/>
                <xsd:element name="EmployeeType" type="xsd:string"/>
                <xsd:element name="Salary" type="xsd:long"/>
            </xsd:sequence>
        </xsd:complexType>
    </xsd:element>
</xsd:schema>
\end{lstlisting}\end{code}

Можно заметить, что в ней описывается сущность \path{Employee} с пятью полями: строковыми
\path{SSN}, \path{Name}, \path{EmployeeType}, целочисленным \path{Salary}
и одно поле с датой~--- \path{DateOfBirth}.

Подключение схемы в программу на языке Gosu происходит с помощью добавления директории с файлом XML схемы в переменную окружения \path{classpath}.
После этого, класс \path{Employee} загружается и становится доступным для использования.

\begin{code}
На следующем примере можно видеть создание экземпляра этого класса и заполнение его полей необходимыми данными:

\begin{lstlisting}[caption={Содержимое файла \texttt{test\_project.gsp} из примера~\ref{gosu-project-layout}.}, label=gosu-xsd-loader]
classpath "../src"
uses xsds.employee.Employee
uses gw.xml.date.XmlDate
var x = new Employee() {
    :SSN = "1011001",
    :Name = "Ted",
    :DateOfBirth = new XmlDate(),
    :EmployeeType = "None",
    :Salary=100000
}
x.print()
\end{lstlisting}

Программа выведет в консоль следующую информацию:
\begin{lstlisting}[language=xml, caption={Результат работы программы \ref{gosu-xsd-loader}.}, label=gosu-xsd-loader-result]
<?xml version="1.0"?>
<Employee>
  <SSN>1011001</SSN>
  <Name>Ted</Name>
  <DateOfBirth>2012-02-09</DateOfBirth>
  <EmployeeType>None</EmployeeType>
  <Salary>100000</Salary>
</Employee>
\end{lstlisting}

\end{code}


\item[Загрузчик типов для создания объектно-реляционного отображения.] Другим примером может служить одна из частей каркаса для построения веб-приложений (web application framework), написанного на языке Gosu~--- Ronin\footnote{\url{http://ronin-web.org/}}.

А именно та часть, которая позволяет получать объектно-реляционное отображение через драйвер JDBC~\cite{jdbc-book} посредством загрузчиков типов, доступных в языке Gosu.

Таблицы отображаются в типы, а столбцы~--- в свойства; при этом отсутствует генерация кода для взаимодействия с базой данных и предоставления объектно-реляционного отображения с помощью автоматически созданных <<классов-оберток>>; также удается избежать многословной настройки соединения, привычной для программ на языке Java.

Вышеперечисленные плюсы позволяют создавать веб-приложения эффективным способом. Существенным недостатком такого стека разработки является невозможность представления программы в бинарном виде (ограничение обусловлено реализацией компилятора языка Gosu). Но для большинства веб-приложений, такая лимитация не является существенной.

\end{description}

Открытая система типов и, в частности, загрузчики типов появились в ответ на необходимость компании Guidewire Software (разработчика языка Gosu) оказывать коммерческую поддержку большому числу заказчиков.
В следствии чего, возникла потребность в извлечении информации из разных форматов данных.

Популярными протоколами представления и обмена данными в мире коммерческой разработки являются форматы XSD и WSDL. Поэтому загрузчики для этих источников информации являются частью стандартной поставки компилятора и среды разработки языка Gosu.

\subsection{Схожие механизмы}\label{similar-mechanisms}
Сходными механизмами для расширения системы типов можно считать подключаемую систему типов (pluggable type system)~\cite{bracha}, а также различные легковесные механизмы, когда новая система типов может быть построена поверх существующей.

Авторы данного подхода предлагают создавать систему типов, не втроенную в язык программирования, а опциональную и не зависящую от конкретной реализации языка программирования.

В таком случае, возможен одновременный (или последовательный) семантический анализ одной и той же программы несколькими реализациями системы типов. Авторами утверждается, что подобный подход может решить некоторые недостатки <<обязательной>> системы типов, без добавления новых.

Подобные механизмы могут быть реализованы как в виде пакетов на базе существующих языков~--- JavaCOP~\cite{javacop2010}, Checker Framework~\cite{checkerframework2008},
так и в виде новых языков программирования~--- Strongtalk~\cite{strongtalk1993}, Newspeak~\cite{newspeak2008}.

Также стоит отметить, что сходными с загрузкой типов, проблемами занимались разработчики языка EL1~\cite{el1} еще в середине 70-x годов XX века.
В результате, язык получил механизмы представления и преобразования типов, которые и позволяли получать эффективно скомпилированный код и компактное представление данных.