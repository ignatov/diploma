\section{Обзор средств расширения систем типов}
В данной главе в разделе~\ref{f-sharp-type-providers} рассматривается механизм поставщиков типов языка F\#~3.0. Далее, в разделе~\ref{gosu-type-loaders} рассказывается про открытую систему типов и механизм загрузчиков типов языка Gosu. А в разделе~\ref{similar-mechanisms} приводится обзор сходных механизмов расширения типов языков программирования. 

\subsection{Поставщики типов в языке F\#}\label{f-sharp-type-providers}
Одновременно с выпуском версии 3.0 языка F\# его разработчики сообщили о введении новой концепции развития~--- информационно насыщенном программировании (Information Rich Programming)~\cite{joepamer2011}.
Одним из нововведений, поддерживающим эту концепцию являются поставщики типов (Type Providers).
Поставщики типов позволяют получить статически типизированный доступ к различным источникам данных.
В стандартной поставке языка F\# существуют реализации поставщиков типов для работы с базами данных, с серверами, поддерживающими протокол OData (Open Data Protocol) и другие\footnote{Информация о поставщиках типов доступна по адресу \url{http://msdn.microsoft.com/en-us/library/hh156509(v=vs.110).aspx}.}.

\subsection{Загрузчики типов в языке Gosu}\label{gosu-type-loaders}
В языке Gosu~\cite{gosuguide} существует открытая система типов (The Open Type System), которая позволяет реализовывать собственные загрузчики типов (type loaders) для любых источников.
Существенным минусом этой реализации является создание типов загружаемых данных на этапе выполнения программы, а не на этапе компиляции.
В стандартной библиотеке языка Gosu существуют загрузчики для получения типов данных из XSD, WSDL и т.д.

\subsection{Схожие механизмы}\label{similar-mechanisms}
Сходными механизмами для расширения системы типов можно считать подключаемую систему типов (pluggable type system)~\cite{bracha}, а также различные легковесные механизмы, когда новая система типов может быть построена поверх существующей.

Подобные механизмы могут быть реализованы как в виде пакетов на базе существующих языков~--- JavaCOP~\cite{javacop2010}, Checker Framework~\cite{checkerframework2008},
так и в виде новых языков программирования~--- Strongtalk~\cite{strongtalk1993}, Newspeak~\cite{newspeak2008}.
