\section{Введение}

Программые интерфейсы, которые используются в прикладных программах, могут иметь различную природу. 

Наиболее традиционным подходом являются библиотеки: они предоставляют набор классов и функций, которые и определяют программный интерфейс с точки зрения используемого языка программирования.
Важным свойством такого решения является то, что компилятор языка программирования может убедиться, что использование интерфейса происходит должным образом с помощью статической проверки типов. Использование библиотеки прекрасно подходит, если интерфейс является неизменным (или изменяется медленно).

Но есть и внешние по отношению к языку протоколы, такие как XSD, WSDL, Protobuf или схемы баз данных. Для таких интерфейсов компилятор не может дать никаких статических гарантий.

Компромиссным решением является кодогенерация вспомогательных классов на используемом языке программирования, которые и используются при доступе к необходимому интерфейсу.

Но разве не здорово было бы иметь язык, который понимает внешние по отношению к нему интерфейсы и предоставляет к ним статически типизированный доступ, без создания каких-либо промежуточных <<классов-оберток>>?

Таким решением проблемы стали загрузчики типов. Это специальные механизмы, которые могут использовать практически любые описания программных интерфейсов (WSDL, XSD, XMI, Ecore, Protobuf, или даже заголовочный файл на языке C) и предоставлять в ним доступ, как будто бы это есть классы на используемом \td языке программирования. Таким образом, получается безопасный доступ к данным и полную синхронизацию с внешним программным интерфейсом.

Примеры таких механизмов появились в языках F\#~3.0~\cite{joepamer2011} и Gosu~\cite{gosuguide}, в данной работе предсатвлен прототип загрузчика типов для языка Extensible Kotlin.

% В современном мире растет количество источников структурированной информации. Для получения доступа к таким источникам из прикладных программ обычно необходимо использовать специальный программный интерфейс (Application Programming Interface, API).

% Хорошо, когда такой интерфейс предоставляется внешней библиотекой, но во многих случаях такой библиотеки может и не быть: например, необходимо получать информацию по нестандартному протоколу или \td.

% В таких случаях перед прикладным программистом встает задача реализаци такого программного интерфейса для нового протокола. Для некоторых случаев такие интерфейсы можно создавать автоматически. Одним из ответов на эту потребность являются механизмы загрузки типов. \td

% В некоторых современных языках программирования (например F\#~3.0 и Gosu) существует поддержка механизмов автоматического создания статически типизированных программных интерфейсов к различным источникам информации.

\clearpage