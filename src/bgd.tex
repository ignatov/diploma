% ********** BGD Chapter **********
\section{Охрана труда}
\label{sec:bgd}
\tolerance=600 Проведение научно-исследовательской деятельности в рамках дипломной работы не мыслимо без использования вычислительной техники. Следовательно, такая деятельность сопряжена с неизбежными рисками. Безопасная организация труда в первую очередь позволяет избежать риска для здоровья и повысить эффективность деятельности.\par\tolerance=200

Охрана труда представляет собой систему законодательных актов, социально"=экономических, организационных, технических и лечебно-профилактических мероприятий и средств, обеспечивающих безопасность, сохранение здоровья и работоспособности человека в процессе труда~\cite{BGDEvd1993}. Охрана труда выявляет и изучает возможные причины производственных несчастных случаев, профессиональных заболеваний, аварий, взрывов, пожаров и разрабатывает систему мероприятий и требований с целью устранения этих причин и создания, безопасных и благоприятных для человека условий труда. Полностью безопасных и безвредных производственных процессов не существует. Задача охраны труда --- свести к минимуму вероятность поражения или заболевания работающего с одновременным обеспечением комфорта при максимальной производительности труда.

\subsection{Организация трудовой деятельности}
\label{sec:bgd:trud}
Задача организации трудовой деятельности с использованием вычислительной техники существенно усложнена тем фактом, что вычислительные машины зачастую сосредоточены в вычислительных центрах (ВЦ). Сложность стоящих перед охраной труда задач требует использования достижений и выводов многих научных дисциплин, прямо или косвенно связанных с задачами создания здоровых и безопасных условий труда. Так как главным объектом охраны труда является человек в процессе труда, то при разработке требований производственной санитарии используются результаты исследований ряда медицинских и биологических дисциплин. Особо тесная связь существует между охраной труда, научной организацией труда, инженерной психологией, технической эстетикой и эргономикой.

\subsubsection{Эргономика}
\label{sec:bgd:trud:ergo}
Эргономика изучает трудовую деятельность в комплексе, в ней объединяются научные дисциплины, развивавшиеся прежде независимо друг от друга.
Эргономика - научная дисциплина, изучающая трудовые процессы с целью создания оптимальных условий труда, что способствует увеличению его производительности, а также обеспечивает необходимые удобства и сохраняет силы, здоровье и работоспособность человека. В последние годы много новых идей возникло в связи с рассмотрением трудовой деятельности как процесса взаимодействия человека с машиной и более сложными системами управления. В связи с этим эргономику условно можно разделить на три составляющие.

\begin{compactitem}
\item Микро-эргономика --- исследование и проектирование систем <<человек--машина>>. Сюда же включаются интерфейсы <<человек--компьютер>> (компьютер рассматривается как часть машины) --- как аппаратные интерфейсы, так и программные. Соответственно, <<эргономика программного обеспечения>> --- это подраздел микро-эргономики. Сюда же относятся системы: <<человек--компьютер--человек>>, <<человек--компьютер--процесс>>, <<человек--программа, ПО, ОС>>.
\item Миди-эргономика --- исследование и проектирование систем <<человек--рабочая группа, коллектив, экипаж, организация>>, <<коллектив--машина>>, <<человек--сеть, сетевое сообщество>>, <<коллектив--организация>>. Сюда входит и проектирование организаций, и планирование работ, и обитаемость рабочих помещений, и гигиена труда, и проектирование залов с дисплеями общего пользования, проектирование интерфейсов сетевых программных продуктов, и многое, многое другое. Исследуется взаимодействие на уровне рабочих мест и производственных задач.
\item Макро-эргономика --- исследование и проектирование систем <<человек--социум, общество, государство>>, <<организация--система организаций>>.
\end{compactitem}

\subsubsection{Психо-физиологические факторы}
\label{sec:bgd:trud:psych}
В комплексе мероприятий по совершенствованию организации труда важная роль отводится внедрению научно-обоснованных режимов труда и отдыха, улучшению условий труда.
Основная цель рационального труда и отдыха --- поддержание работоспособности на оптимальном уровне. Необходимость чередования труда и отдыха обусловлена физиологическими закономерностями и играет большую роль в поддержании трудового ритма.
Работоспособность работника в течение рабочего дня не является величиной стабильной. Основные фазы работоспособности:
\begin{compactitem}
\item вырабатывание и нарастающая работоспособность;
\item высокая, устойчивая работоспособность;
\item падение работоспособности в результате развивающегося утомления.
\end{compactitem}
Оптимальный режим труда и отдыха должен включать паузы. При неблагоприятных условиях труда высокий уровень работоспособности составляет не менее 75\% рабочего времени. Период вырабатывания составляет не более 40 минут, а восстановительный период --- не более 10--15 минут.
Наибольшая работоспособность инженерно-технических работников наблюдается с 10 до 12 и с 16 до 18 часов. Рекомендуется делать перерывы по 8--10 минут каждые 2 часа в первой половине дня и 5--8 минут через каждый час во второй половине дня.

\subsubsection{Рабочее место}
\label{sec:bgd:trud:workplace}
Под рабочим местом пользователя понимается не только стол, а пространство, где находится и работает человек, оснащенное необходимыми техническими средствами, в котором совершается трудовая деятельность. Организацией рабочего места называется система мероприятий по оснащению рабочего места средствами и предметами труда и их размещению в определенном порядке.

В соответствии с требованиями эргономики, рабочее место должно быть приспособлено для конкретного вида деятельности и для работников определенной квалификации с учетом их физической и психических возможностей и особенностей. Конструкция рабочего места должна обеспечивать быстроту, безопасность, простоту и экономичность технического обслуживания в нормальных и аварийных условиях; полностью отвечать функциональным требованиям и предполагаемым условиям эксплуатации. При конструировании производственного оборудования необходимо предусматривать возможность регулирования отдельных его элементов с тем, чтобы обеспечивать оптимальное положение работающего. При организации рабочего места учитываются также антропометрические данные пользователя.

Схемы размещения рабочих мест с видеодисплейными терминалами (ВДТ) и персональными электронно-вычислительными машинами (ПЭВМ) должны учитывать расстояния между рабочими столами с видеомониторами (в направлении тыла поверхности одного видеомонитора и экрана другого видеомонитора), которое должно быть не менее 2,0 м, а расстояние между боковыми поверхностями видеомониторов --- не менее 1,2 м.

Рабочие места с ВДТ и ПЭВМ в залах электронно-вычислительных машин или в помещениях с источниками вредных производственных факторов должны размещаться в изолированных кабинах с организованным воздухообменом. Рабочие места с ВДТ и ПЭВМ при выполнении творческой работы, требующей значительного умственного напряжения или высокой концентрации внимания, следует изолировать друг от друга перегородками высотой 1,5--2,0 м.

Оконные проемы в помещениях использования ВДТ и ПЭВМ должны быть оборудованы регулируемыми устройствами типа жалюзи, занавесей и др.

Шкафы, сейфы, стеллажи для хранения дисков, дискет, комплектующих деталей, запасных блоков ВДТ и ПЭВМ, инструментов следует располагать в подсобных помещениях, для учебных заведений --- в лаборантских. При отсутствии подсобных помещений или лаборантских допускается размещение шкафов, сейфов и стеллажей в помещениях непосредственного использования ВДТ и ПЭВМ при соблюдении требований к площади помещений и требований.

Конструкция рабочего стола должна обеспечивать оптимальное размещение на рабочей поверхности используемого оборудования с учетом его количества и конструктивных особенностей (размер ВДТ и ПЭВМ, клавиатуры, пюпитра и др.). При этом допускается использование рабочих столов различных конструкций, отвечающих современным требованиям эргономики. Покрытие стола должно быть матовым (с коэффициентом отражения 20--50\%) и легко чиститься; углы и передняя грань столешницы должны быть закругленными. Параметры стола указаны в таблице~\ref{tab:table_height}.
\begin{table}\begin{center}
\caption{Высота рабочей поверхности стола для пользователей}
\label{tab:table_height}
\begin{tabular}{|c|c|c|}
\hline
\multirow{3}{2.5cm}{Рост пользователя в обуви (см)}
& \multicolumn{2}{|c|}{Высота над полом (мм)} \\
\cline{2-3}
&\multirow{2}{2.5cm}{поверхность стола}&\multirow{2}{4cm}{пространство для ног, не менее}\\
&&\\
%& поверхность & пространство для \\
%& стола       & ног, не менее \\
\hline
116--130 & 520 & 400 \\
\hline
131--145 & 580 & 520 \\
\hline
146--160 & 640 & 580 \\
\hline
161--175 & 700 & 640 \\
\hline
выше 175 & 760 & 700 \\
\hline
\end{tabular}\end{center}\end{table}

Рабочий стол должен иметь пространство для ног высотой не менее той, которая указана в таблице~\ref{tab:table_height}, шириной --- не менее 500 мм, глубиной на уровне колен --- не менее 450 мм и на уровне вытянутых ног --- не менее 650 мм.

Клавиатуру следует располагать на поверхности стола на расстоянии 100--300 мм от края, обращенного к пользователю, или на специальной регулируемой по высоте рабочей поверхности, отделенной от основной столешницы.

Конструкция рабочего стула (кресла) должна обеспечивать поддержание рациональной рабочей позы при работе на ВДТ и ПЭВМ, позволять изменять позу с целью снижения статического напряжения мышц шейно-плечевой области и спины для предупреждения развития утомления. Тип рабочего кресла должен выбираться в зависимости от характера и продолжительности работы с учетом роста пользователя (таблица~\ref{tab:chair}).
\begin{table}[tb]\begin{center}
\caption{Параметры рабочего кресла}
\label{tab:chair}
\begin{tabular}{|p{6cm}|c|c|c|c|c|}
\hline
\multirow{2}{6cm}{Параметры стула} & \multicolumn{5}{|c|}{Рост пользователя в обуви (см)}\\\cline{2-6}
                               & 116-130 & 131-145 & 146-160 & 161-175 & более 175 \\\hline
Высота сиденья над полом (мм)  & \strut 300     & 340     & 380     & 420     & 460 \\\hline
Ширина сиденья, не менее (мм)  & 270     & 290     & 320     & 340     & 360 \\\hline
Глубина сиденья (мм)           & 290     & 330     & 360     & 380     & 400 \\\hline
Высота нижнего края спинки над сиденьем (мм)
             & \tevc{2}{130} & \tevc{2}{150} & \tevc{2}{160} & \tevc{2}{170} & \tevc{2}{190} \\\hline
Высота верхнего края спинки над сиденьем (мм)
             & \tevc{2}{280} & \tevc{2}{310} & \tevc{2}{330} & \tevc{2}{360} & \tevc{2}{400} \\\hline
Высота линии прогиба спинки, не менее (мм)
             & \tevc{2}{170} & \tevc{2}{190} & \tevc{2}{200} & \tevc{2}{210} & \tevc{2}{220} \\\hline
Радиус изгиба переднего края сиденья (мм) & \multicolumn{5}{|c|}{\tevc{2}{20-50}} \\\hline
Угол наклона сиденья (град) & \multicolumn{5}{|c|}{0-4} \\\hline
Угол наклона спинки (град) & \multicolumn{5}{|c|}{95-108} \\\hline
Радиус спинки в плане, не менее (мм) & \multicolumn{5}{|c|}{\tevc{2}{300}} \\\hline
\end{tabular}\end{center}\end{table}
При длительной работе кресло должно быть массивным, при кратковременной --- легкой конструкции, свободно отодвигающееся.

Рабочее кресло должен быть подъемно-поворотным и регулируемым по высоте и углам наклона сиденья и спинки, а также расстоянию спинки от переднего края сиденья, при этом регулировка каждого параметра должна быть независимой, легко осуществляемой и иметь надежную фиксацию.

Поверхность сиденья, спинки и других элементов кресла должна быть полумягкой, с нескользящим, неэлектризующимся и воздухопроницаемым покрытием, обеспечивающим легкую очистку от загрязнений.
Визуальные эргономические параметры ВДТ являются параметрами безопасности, и их неправильный выбор приводит к ухудшению здоровья пользователей. Все ВДТ должны иметь гигиенический сертификат, включающий, в том числе, оценку визуальных параметров.

Конструкция ВДТ, его дизайн и совокупность эргономических параметров должны обеспечивать надежное и комфортное считывание отображаемой информации в условиях эксплуатации. Устройство ВДТ должно обеспечивать возможность фронтального наблюдения экрана путем поворота корпуса в горизонтальной плоскости вокруг вертикальной оси в пределах 30 градусов и в вертикальной плоскости вокруг горизонтальной оси в пределах 30 градусов с фиксацией в заданном положении. Дизайн ВДТ должен предусматривать окраску корпуса в спокойные мягкие тона с диффузным рассеиванием света. Корпус ВДТ и ПЭВМ, клавиатура и другие блоки и устройства ПЭВМ должны иметь матовую поверхность одного цвета с коэффициентом отражения 0,4--0,6 и не иметь блестящих деталей, способных создавать блики. На лицевой стороне корпуса ВДТ не рекомендуется располагать органы управления, маркировку, какие-либо вспомогательные надписи и обозначения. При необходимости расположения органов управления на лицевой панели они должны закрываться крышкой или быть утоплены в корпусе.

Для обеспечения надежного считывания информации при соответствующей степени комфортности ее восприятия должны быть определены оптимальные и допустимые диапазоны визуальных эргономических параметров. Визуальные эргономические параметры ВДТ и пределы их изменений, в которых должны быть установлены оптимальные и допустимые диапазоны значений, приведены в таблице~\ref{tab:vdt_params}.
\begin{table}[b]\begin{center}
\caption{Допустимые визуальные эргономические параметры}
\label{tab:vdt_params}
\begin{tabular}{|p{7cm}|c|c|}\hline
\multirow{2}{7cm}{Наименование параметров} & \multicolumn{2}{|c|}{Пределы значений параметров}\\\cline{2-3}
                               & миним. (не менее) & макс. (не более) \\\hline
Яркость знака (яркость фона), измеренная в темноте (кд/м2) & \tevc{2}{35} & \tevc{2}{120} \\\hline
Внешняя освещенность экрана (лк) & 100 & 250 \\\hline
Угловой размер знака (угл. мин.) & 16 & 60 \\\hline
\end{tabular}\end{center}
\tabannot
угловой размер знака - угол между линиями, соединяющими крайние точки знака по высоте и глаз наблюдателя. Угловой размер знака определяется по формуле: $a=\arctg(h/2l)$, где $h$ --- высота знака, $l$ --- расстояние от знака до глаза наблюдения.
\end{table}

При проектировании и разработке ВДТ сочетания визуальных эргономических параметров и их значения, соответствующие оптимальным и допустимым диапазонам, полученные в результате испытаний в специализированных лабораториях, аккредитованных в установленном порядке, и подтвержденные соответствующими протоколами, должны быть внесены в техническую документацию на ВДТ. При отсутствии в технической документации на ВДТ данных об оптимальных и допустимых диапазонах значений эргономических параметров эксплуатации ВДТ не допускается.

Конструкция ВДТ должна предусматривать наличие ручек регулировки яркости и контраста, обеспечивающих возможность регулировки этих параметров от минимальных до максимальных значений.

В целях обеспечения защиты от электромагнитных и электростатических полей допускается применение экранных фильтров, специальных экранов и других средств индивидуальной защиты, прошедших испытания в аккредитованных лабораториях и имеющих соответствующий гигиенический сертификат. Допустимые значения параметров излучений указаны в таблице~\ref{tab:vdt_em}.
\begin{table}[tb]\begin{center}
\caption{Допустимые значения параметров неионизирующих электромагнитных излучений}
\label{tab:vdt_em}
\begin{tabular}{|p{10cm}|c|c|}\hline
Наименование параметров & Допустимое значение \\\hline
Напряженность электромагнитного поля по электрической составляющей на расстоянии 50 см от    видеомонитора & \tevc{3}{10 В/м} \\\hline
Напряженность электромагнитного поля по магнитной составляющей на расстоянии 50 см от поверхности видеомонитора & \tevc{3}{0,3 А/м} \\\hline
Напряженность электростатического поля не должна превышать: & \\\hline
для взрослых пользователей & 20 кВ/м \\\hline
для детей дошкольных учреждений и учащихся средних специальных и высших учебных заведений & \tevc{2}{15 кВ/м} \\\hline
Напряженность электромагнитного поля на расстоянии 50 см вокруг ВДТ по электрической составляющей должна быть не более: & \\\hline
 в диапазоне частот 5 Гц -- 2 кГц & 25 В/м \\\hline
 в диапазоне частот 2--400 кГц & 2,5 В/м \\\hline
Плотность магнитного  потока  должна быть не более: & \\\hline
 в диапазоне частот 5 Гц -- 2 кГц & 250 нТл \\\hline
 в диапазоне частот 2--400 кГц & 25 нТл \\\hline
Поверхностный электростатический потенциал не должен превышать & \tevc{2}{500 В} \\\hline
\end{tabular}\end{center}\end{table}

Мощность экспозиционной дозы рентгеновского излучения в любой точке на расстоянии 0,05 м от экрана и корпуса ВДТ, при любых положениях регулировочных устройств не должна превышать 100 мкР/ч.

Конструкция клавиатуры должна предусматривать:
\begin{compactitem}
\item исполнение в виде отдельного устройства с возможностью свободного перемещения;
\item опорное приспособление, позволяющее изменять угол наклона поверхности клавиатуры в пределах от 5 до 15 градусов;
\item высоту среднего ряда клавиш не более 30 мм;
\item расположение часто используемых клавиш в центре, внизу и справа, редко используемых --- вверху и слева;
\item выделение цветом, размером, формой и местом расположения функциональных групп клавиш;
\item минимальный размер клавиш --- 13 мм, оптимальный --- 15 мм;
\item клавиши с углублением в центре и шагом 19 $\pm$ 1 мм;
\item расстояние между клавишами не менее 3 мм;
\item одинаковый ход для всех клавиш с минимальным сопротивлением нажатию 0,25 Н и максимальным --- не более 1,5 Н;
\item звуковую обратную связь от включения клавиш с регулировкой уровня звукового сигнала и возможностью ее отключения.
\end{compactitem}

При работе за ПЭВМ очень важно сохранять правильную осанку, при которой позвоночник будет отдыхать, а не напрягаться. В этом помогает хорошо подобранное рабочее кресло. Спинка кресла должна поддерживать нижнюю половину спины, но при этом не быть жестко закрепленной, чтобы не препятствовать движениям в процессе работы. Ноги должны большую часть времени стоять на полу полной ступней, согнуты чуть больше, чем под прямым углом.

Голова должна быть немного наклонена вперед, это наиболее естественное состояние. Монитор необходимо установить так, чтобы расстояние от глаз пользователя до любой точки монитора было примерно одинаковое и составляло 50--70 см.

При работе на клавиатуре руки не должны находиться на весу, а опираться на подлокотники. Клавиатура обязательно должна располагаться чуть ниже локтя. Угол, образуемый между плечом и предплечьем, должен составлять около 120 градусов.

\subsection{Помещение}
\label{sec:bgd:pomeschenie}
\tolerance=600 Помещения с ВДТ и ПЭВМ должны иметь естественное и искусственное освещение~\cite{BGDGigi2004}. Естественное освещение должно осуществляться через светопроемы, ориентированные преимущественно на север и северо-восток.\par\tolerance=200

Расположение рабочих мест с ВДТ и ПЭВМ для взрослых пользователей в подвальных помещениях не допускается. Размещение рабочих мест с ВДТ и ПЭВМ во всех учебных заведениях и дошкольных учреждениях не допускается в цокольных и подвальных помещениях. В случаях производственной необходимости эксплуатация ВДТ и ПЭВМ в помещениях без естественного освещения может проводиться только по согласованию с органами и учреждениями Государственного санитарно-эпидемиологического надзора.
Площадь на одно рабочее место с ВДТ или ПЭВМ для взрослых пользователей должна составлять не менее 6,0 кв. м, а объем --- не менее 20,0 куб. м.

Производственные помещения, в которых для работы используются преимущественно ВДТ и ПЭВМ (диспетчерские, операторские, расчетные и др.), и учебные помещения (аудитории вычислительной техники, дисплейные классы, кабинеты и др.) не должны граничить с помещениями, в которых уровни шума и вибрации превышают нормируемые значения (механические цеха, мастерские, гимнастические залы и т.п.).

Звукоизоляция ограждающих конструкций помещений с ВДТ и ПЭВМ должна отвечать гигиеническим требованиям и обеспечивать нормируемые параметры шума.

Помещения с ВДТ и ПЭВМ должны быть оборудованы системами отопления, кондиционирования воздуха или эффективной приточно-вытяжной вентиляцией. Расчет воздухообмена следует проводить по избыткам тепла от машин, людей, солнечной радиации и искусственного освещения. Нормируемые параметры микроклимата, ионного состава воздуха, содержание вредных веществ в нем должны отвечать требованиям Санитарных правил.

Для внутренней отделки интерьера помещений с ВДТ и ПЭВМ должны использоваться диффузно-отражающие материалы с коэффициентом отражения для потолка --- 0,7--0,8; для стен --- 0,5--0,6; для пола --- 0,3--0,5. Полимерные материалы, используемые для внутренней отделки интерьера помещений с ВДТ и ПЭВМ, должны быть разрешены для применения органами и учреждениями государственного санитарно-эпидемиологического надзора. В дошкольных и всех учебных учреждениях, включая вузы, запрещается для отделки интерьера помещений с ВДТ и ПЭВМ применять полимерные материалы (древесностружечные плиты, слоистый бумажный пластик, синтетические ковровые покрытия др.), выделяющие в воздух вредные химические вещества.

Поверхность пола в помещениях эксплуатации ВДТ и ПЭВМ должна быть ровной, без выбоин, нескользкой, удобной для очистки и влажной уборки, обладать антистатическими свойствами.

\subsection{Освещение}
\label{sec:bgd:lighting}
Рабочие места с ВДТ и ПЭВМ по отношению к световым проемам должны располагаться так, чтобы естественный свет падал сбоку, преимущественно слева.

Искусственное освещение в помещениях эксплуатации ВДТ и ПЭВМ должно осуществляться системой общего равномерного освещения. В производственных и административно-общественных помещениях, в случаях преимущественной работы с документами, допускается применение системы комбинированного освещения.

Освещенность на поверхности стола в зоне размещения рабочего документа должна быть 300--500 лк. Допускается установка светильников местного освещения для подсветки документов. Местное освещение не должно создавать бликов на поверхности экрана и увеличивать освещенность экрана более 300 лк.

В качестве источников света при искусственном освещении должны применяться преимущественно люминесцентные лампы, так как у них высокая светоотдача (до 120 лм/Вт и более), продолжительный срок службы (до 10 000 ч.), малая яркость светящейся поверхности, близкий к естественному спектральный состав излучаемого света, что обеспечивает хорошую светопередачу. Допускается применение ламп накаливания в светильниках местного освещения. Светильники местного освещения должны иметь непросвечивающий отражатель с защитным углом не менее 40 градусов.

Общее освещение следует выполнять в виде сплошных или прерывистых линий светильников, расположенных сбоку от рабочих мест, параллельно линии зрения пользователя при рядном расположении ВДТ и ПЭВМ. При периметральном расположении компьютеров линии светильников должны располагаться локализовано над рабочим столом ближе к его переднему краю, обращенному к оператору.

Для обеспечения нормируемых значений освещенности в помещениях использования ВДТ и ПЭВМ следует проводить чистку стекол оконных рам и светильников не реже двух раз в год и проводить своевременную замену перегоревших ламп.

\subsection{Микроклимат}
\label{sec:bgd:microclimat}
Микроклиматические условия устанавливаются ГОСТом 12.1.005-88~\cite{BGDGost_12_1_005_88} и СанПиНом 2.2.2/2.4.1340-03~\cite{BGDSanpin2_2_2_2_4_1340_03}. Оптимальные и допустимые параметры микроклимата приведены в таблице~\ref{tab:microclimat} и таблице~\ref{tab:microclimat_dopusk}, соответственно.
\begin{table}\begin{center}
\caption{Оптимальные параметры микроклимата}
\label{tab:microclimat}
\begin{tabular}{|p{3cm}|p{2cm}|p{2.5cm}|p{3cm}|p{2.5cm}|}\hline
\multirow{3}{3cm}{Период года} & \multirow{3}{2cm}{Категория работ} & \multirow{3}{2.5cm}{Температура (\textdegree)} & \multirow{3}{3cm}{Относительная влажность (\%)} & \multirow{3}{2.5cm}{Скорость движения воздуха (м/с)} \\
&&&&\\&&&&\\\hline
\multirow{2}{3cm}{Холодный и переходный}
   & Легкая 1а & \tehc{22--24} & \tehc{40--60} & \tehc{0.1} \\\cline{2-5}
   & Легкая 1б & \tehc{21--23} & \tehc{40--60} & \tehc{0.1} \\\hline
\multirow{2}{3cm}{Теплый}
   & Легкая 1а & \tehc{23--25} & \tehc{40--60} & \tehc{0.1} \\\cline{2-5}
   & Легкая 1б & \tehc{22--24} & \tehc{40--60} & \tehc{0.2} \\\hline
\end{tabular}\end{center}
\tabannot
\begin{compactitem}
\item к категории 1а относятся работы, производимые сидя и не требующие физического напряжения, при которых расход энергии составляет до 120 ккал/ч;
\item к категории 1б относятся работы, производимые сидя, стоя или связанные с ходьбой и сопровождающиеся некоторым физическим напряжением, при которых расход энергии составляет от 120 до 150 ккал/ч.
\end{compactitem}
\medskip
\end{table}

\begin{table}[h!]\begin{center}
\caption{Допустимые параметры микроклимата}
\label{tab:microclimat_dopusk}
\begin{tabular}{|p{3cm}|p{2cm}|p{4cm}|p{3cm}|p{2.5cm}|}\hline
\multirow{3}{3cm}{Период года} & \multirow{3}{2cm}{Категория работ} & \tevc{3}{Температура (\textdegree)} & \multirow{3}{3cm}{\raggedright Относительная влажность (\%)} & \multirow{3}{2.5cm}{\raggedright Скорость движения воздуха (м/с)} \\
&&&&\\&&&&\\\hline
\tehc{Холодный и переходный} & \multirow{2}{2cm}{Легкая} & \tevc{2}{19--25} & \tevc{2}{не более 75} & \tevc{2}{< 0,2} \\\hline
\multirow{5}{3cm}{Теплый} & \multirow{5}{2cm}{Легкая} & не более чем на 3\textdegree\ выше средней температуры наружного воздуха в 13 ч. самого жаркого месяца, но не более 28\textdegree &
\smallskip\tevc{5}{$\begin{aligned}24^\circ \textup{C: }<75\\
                                   25^\circ \textup{C: }<70\\
                                   26^\circ \textup{C: }<65\\
                                   27^\circ \textup{C: }<60\\
                                   28^\circ \textup{C: }<55\end{aligned}$} & \tevc{5}{< 0,2--0,5} \\\hline
\end{tabular}
\onelineskip
\end{center}\end{table}

\subsection{Уровень шума}
\label{sec:bgd:noise}
На рабочем месте пользователя источником шума является вычислительная машина, производящая постоянный шум. Шум предоставляет собой сочетание звуков, различных по интенсивности и частоте в частотном диапазоне 10--20 кГц, не несущих полезной информации. Шум вредно воздействует не только на органы слуха, но и на весь организм человека в целом через центральную нервную систему. Шум - причина преждевременного утомления, ослабления внимания, памяти. Во всех учебных и дошкольных помещениях с ВДТ и ПЭВМ уровень шума на рабочем месте не должен превышать 50 дБА. Шумящее оборудование (принтеры, сканеры, факсы и т.д.), уровни шума которого превышают нормированные, должно находиться вне помещения с ВДТ и ПЭВМ.

Снизить уровень шума в помещениях с ВДТ и ПЭВМ можно использованием звукопоглощающих материалов с максимальными коэффициентами звукопоглощения в области частот 63--8000 Гц для отделки помещений (разрешенных органами и учреждениями Госсанэпиднадзора России). Дополнительным звукопоглощением служат однотонные занавеси из плотной ткани, гармонирующие с окраской стен и подвешенные в складку на расстоянии 15--20 см от ограждения. Ширина занавеси должна быть в 2 раза больше ширины окна.

\subsection{Электробезопасность}
\label{sec:bgd:electricity}
Электробезопасность --- система организационно-технических мероприятий и средств, обеспечивающих защиту людей от вредного и опасного воздействия электрического тока, электрической дуги, электромагнитного поля и статического электричества.

По степени поражения людей электрическим током рабочее помещение программиста относится к категории помещений без повышенной опасности, поскольку является сухим, нежарким, непыльным, с нетокопроводящим полом, возможность случайного одновременного прикосновения к токоведущим частям и заземленным конструкциям присутствует лишь в случае грубейшего нарушения техники безопасности (ТБ) при техническом обслуживании ЭВМ и периферийных устройств. В связи с этим требуется соблюдение необходимых мер предосторожности, в т. ч. использование средств индивидуальной защиты (при техническом обслуживании оборудования), что закреплено в инструкции по ТБ, в соответствии с ГОСТом 21552-84~\cite{BGDGost_21552-84}. Электропитание осуществляется от однофазной сети переменного тока номинальным напряжением 220 В и частотой переменного тока 50 Гц с заземленной нейтралью. Предельные отклонения по напряжению и частоте должны соответствовать ГОСТу 12.1.038-82~\cite{BGDGost_12_1_038_82}. Для защиты от поражения электрическим током выполняется заземление корпусов оборудования. Все оборудование имеет предохранители в цепи питания. При прикосновении в ВЦ к любому из элементов ПЭВМ могут возникнуть разрядные токи статического электричества. Для снижения возникающих статических зарядов в ВЦ покрытие полов следует выполнять из однослойного поливинилхлоридного линолеума. Не рекомендуется носить одежду из синтетических тканей. К общим мероприятиям защиты от статического электричества в ВЦ можно отнести общее и местное увлажнение воздуха (до 50\%), ионизацию воздуха.

\subsection{Пожаробезопасность}
\label{sec:bgd:fire}
Согласно ГОСТ 12.1.004-91~\cite{BGDGost_12_1_004_91} существуют следующие опасные факторы:
\begin{compactitem}
\item пламя и искры;
\item повышенная температура окружающей среды;
\item токсичные продукты горения и термического разложения;
\item пониженная концентрация кислорода.
\end{compactitem}

Противопожарная защита обеспечивается следующими мерами:
\begin{compactitem}
\item применение средств пожаротушения, установка сигнализации и устройств тушения, ограничивающих распространение пожара, мероприятия по эвакуации людей, наличие средств индивидуальной защиты и средств противодымной защиты;
\item наличие противопожарных перегородок и отсеков, устройств автоматического отключения систем;
\item планировка эвакуационных путей и выходов;
\item оповещение людей;
\item технические средства для эвакуации и спасения людей;
\item наличие огнетушащих веществ.
\end{compactitem}

Пожарную опасность в ВЦ представляют носители информации, поэтому помещение должно быть оборудовано несгораемыми стеллажами и шкафами. Хранение перфокарт, лент, дисков должно производиться в металлических кассетах. Не допускается размещение складских помещений, а также пожаровзрывоопасных производств над и под залами ЭВМ, а также смежных с ними помещениях.

Система вентиляции ВЦ должна быть оборудована устройством, обеспечивающим автоматическое отключение ее при пожаре, а также огнедымозадерживающими устройствами.

Подача воздуха к ЭВМ для охлаждения должна осуществляться по самостоятельному воздуховоду. Присоединение этих воздуховодов к общему коллектору допускается только после огне --- и дымозадерживающих клапанов.

Система электропитания ЭВМ должна иметь блокировку, обеспечивающую отключение ЭВМ в случае остановки системы кондиционирования и охлаждения. Промывка ячеек и других съемных устройств горючими жидкостями допускается только в специальных помещениях, оборудованных проточно-вытяжной системой.

В здании ВЦ должна быть предусмотрена автоматическая пожарная сигнализация. В залах ЭВМ, за подвесными потолками, в хранилищах информации, кладовых запасного оборудования необходимо устанавливать извещатели, реагирующие на дым. Во всех других помещениях ВЦ допускается установка типовых пожарных извещателей.

Для тушения возможных пожаров ВЦ необходимо оборудовать автоматическими установками объемного газового тушения с выводом огнегасительного вещества в кабельные каналы и потоки.

% ********** End of chapter **********
