\documentclass[a4paper,12pt]{article}
\usepackage[utf8]{inputenc}
\usepackage[english,russian]{babel}
% \usepackage{floatflt}
\usepackage{longtable}
\usepackage{amsmath}
\usepackage{amsfonts}
\usepackage{amssymb}
\usepackage{cite}
\usepackage{graphicx}
\usepackage{epstopdf}
\usepackage{datetime}
\usepackage{indentfirst}
\usepackage{subfigure}
\usepackage[section]{placeins}
\usepackage{afterpage}
\usepackage{url}
\usepackage[unicode]{hyperref}
\usepackage{ucs}
\usepackage{listings}
\usepackage{microtype}
\usepackage{paralist}
\usepackage{multirow}
\usepackage{thesis}
\usepackage{color}
\usepackage{xcolor}
\usepackage{float}
\usepackage{bold-extra}

\newcommand{\ssection}[1]{\section*{#1}{\centering #1}}
% \frenchspacing

% \renewcommand{\textfraction}{0.05}
% \renewcommand{\topfraction}{0.95}
% \renewcommand{\bottomfraction}{0.95}
% \renewcommand{\floatpagefraction}{0.35}
% \setcounter{totalnumber}{10}

\usepackage{geometry}
\geometry{left=2cm}
\geometry{right=2cm}
\geometry{top=2cm}
\geometry{bottom=2cm}

\linespread{1.3}

\pagestyle{empty}

\begin{document}
% \ssection{Реферат}
% \begin{center} \Large{\bfseries{Реферат} \end{center}
\begin{center}
\Large\bfseries
Реферат
\end{center}

Пояснительная записка содержит 48 страниц, включая 7 таблиц и 27 листингов с примерами исходных кодов.
\newline\newline
ЗАГРУЗЧИК ТИПОВ, КОМПИЛЯТОР ЯЗЫКА ПРОГРАММИРОВАНИЯ, РАСШИРЯЕМЫЙ КОМПИЛЯТОР, СТАТИЧЕСКАЯ ТИПИЗАЦИЯ, ПРОГРАММНЫЙ ИНТЕРФЕЙС
\newline

Работа состоит из четырех частей. В первой части производится обзор существующих средств для пополнения или синтеза типов прикладной программы из внешних по отношению к компилятору языка программирования источников информации. Во второй части рассказывается о языке Kotlin; об архитектуре расширенных компиляторов; a также о реализации механизма загрузки типов из XML схем и заголовочных файлов на языке C. Третья часть посвящена сравнению полученных результатов с аналогичными решениями, доступными на сегодняшний день. Четвертая часть посвящена охране труда.

Дипломный проект посвящен разработке загрузчиков типов~---
специальных механизмов, которые могут использовать практически любые описания программных интерфейсов (WSDL, XSD или даже заголовочный файл на языке~C) и предоставлять к ним доступ, как будто бы это протокол взаимодействия написан на используемом языке программирования. С помощью такого подхода предоставляется безопасный доступ к данным и полная синхронизация с внешним программным интерфейсом в случае его изменения.

В результате выполнения работы созданы реализации механизмов загрузки типов для работы с XML схемами и заголовочными файлами на языке C в рамках компилятора Extensible Kotlin.
Данный компилятор построен на основе компилятора языка Kotlin~--- нового объектно-ориентированного языка, предназначенного для промышленной разработки приложений и компилируемого в переносимый байт-код для виртуальной машины Java.

Предложенные в работе решения позволяют улучшить некоторые качественные показатели процесса разработки прикладного программного обеспечения, а именно: упростить рабочий процесс программиста и улучшить согласованность с внешними интерфейсами. Дополнительным результатом можно считать повышение таких количественных метрик как объем исходного кода и итоговый размер бинарной сборки.

\end{document}
