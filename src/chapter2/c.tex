\subsection{Загрузчик типов из заголовочных файлов на языке С}
Иногда перед прикладным программистом ставится задача перенесения существующего кода на новую программную платформу.
В ходе таких работ может возникнуть необходимость в вызове кода из библиотек, доставшихся в наследство (legacy code).

Для таких случаев язык Extensible Kotlin предоставляет возможность загрузки структур данных и функций из заголовочных файлов на языке C.
При загрузке, структуры данных преобразуются в классы, а функции~--- в функции.

\subsubsection{Постановка задачи}
\td Необходимо реализовать механизм загрузки типов и функций для программ на языке С. Сравнить полученный механизм с существующими по нескольким характеристикам: количество кода, которое необходимо написать для реализации самого загрузчика и количество кода, которое необходимо написать для его использования.

\subsubsection{Пример входных данных}
Рассмотрим следующий пример. Представим, что у нас есть библиотека на языке C, а также её заголовочный файл со следующим содержанием:
\lstinputlisting[language=C]{listings/point.h}

\subsubsection{Импортирование заголовочного файла на языке C программу на языке Extensible Kotlin}
Аналогично подключению загрузчика типов для схем XML, подключение происходит явно. \td

\subsubsection{Примеры использования}
После подключения загрузчика и импорта загруженных классов и функций можно приступать к вызову кода на языке C из программы на языке Kotlin:
\begin{lstlisting}
import org.libs.point.Point, move, copy_with_offset

fun main(args: Array<String>) {
  val zero: Point = Point(0, 0)
  move(zero, 100, 100)
  val result = copy_with_offset(zero, 100, 100)
  print("x: " + result.x + ", y: " + result.y)
}
\end{lstlisting}
% Стоит отметить, что в данной реализации загрузчика классы-обёртки для взаимодействия с кодом на языке C создаются, упаковываются в конечную сборку и вызываются во время исполнения программы.
\subsubsection{Алгоритм работы загрузчика функций и типов}
\subsubsection{Результаты}