\clearpage
\subsection{Загрузчик типов из схем XML}\label{xml-loader}

В данном разделе подробно описывается реализованный механизм загрузки типов из заголовочных файлов на языке C в компиляторе Extensible Kotlin.

\begin{code}
\subsubsection{Пример входных данных}
Представим, что у нас есть следующая схема, описывающая XML файлы, которые мы хотим считывать в нашей программе:
\lstinputlisting[language=xml, caption={Пример XML cхемы.}, label=xsd-example-input]{listings/shiporder.xsd}
\end{code}

После изучения схемы видно, что у элемента \path{shiporder} есть список дочерних элементов \path{item}, у каждого из которых есть три обязательных поля: строковое \path{title}, положительное целое \path{quantity}, десятичное \path{price} (описывающее стоимость товара) и одно необязательное строковое~--- \path{note}.

Подобные рассмотрения приводят к мысли, что схемы данных XML могут один к одному транслироваться в классы языка программирования и обратно.

Следовательно, имеет смысл написание специального расширения компилятора языка, которое позволит автоматически загружать типы данных из файлов со схемами и оперировать со статически типизированной информацией, которая согласована с этими схемами.

\begin{code}
Пример файла, удовлетворяющей схеме, приведенной выше:
\lstinputlisting[language=xml, caption={Пример XML файла, удовлетворяющего схеме из примера~\ref{xsd-example-input}.}, label=xml-for-xsd-example]{listings/order.xml}
\end{code}

\subsubsection{Примеры языков со встроенной поддержкой XML}
Некоторые современные языки программирования обладают встроенной поддержкой XML.

Например, в уже упомянутом языке Scala~\cite{scala-spec}, литералы XML являются частью синтаксиса:

\begin{code}\begin{lstlisting}[caption={Пример использования литералов XML в языке Scala.}, label=scala-xml-example]
val document =
  <root>
    <child>
      <grandchild an_attribute="value1" />
      <grandchild an_attribute="value2" />
    </child>
  </root>
\end{lstlisting}\end{code}

\begin{code}
Во внутреннем представлении компилятора каждый XML-литерал является экземпляром класса \path{scala.xml.Elem}\footnote{http://www.scala-lang.org/api/current/scala/xml/Elem.html}:

\begin{lstlisting}[caption={Внутреннее представление XML литералов из примера~\ref{scala-xml-example}.}, label={scala-xml-example-internal}]
val document =
scala.xml.Elem(null, "root", scala.xml.Null, scala.xml.TopScope,
  scala.xml.Elem(null, "child", scala.xml.Null, scala.xml.TopScope,
    scala.xml.Elem(null, "grandchild",
      new scala.xml.UnprefixedAttribute(
        "an_attribute","value1", scala.xml.Null),
      scala.xml.TopScope,
      scala.xml.Text("content1")
    ),
    scala.xml.Elem(null, "grandchild",
      new scala.xml.UnprefixedAttribute(
        "an_attribute","value2", scala.xml.Null),
      scala.xml.TopScope,
      scala.xml.Text("content2")
    )
  )
)
\end{lstlisting}\end{code}

В языке Scala для работы с XML предлагается достаточно мощный интерфейс. Именно он и используется в многочисленных каркасах для создания веб-приложений (web application frameworks), написанных на Scala.

С точки зрения подобного класса программ (библиотек для создания веб-приложений), нативная поддержка XML
литералов в языке является несомненным плюсом, нежели минусом, позволяя прозрачно работать с пользовательскими шаблонами.

Также стоит отметить критику пакета \path{scala.xml} о недостаточной производительности, несогласованности некоторых методов и т.п.
Главным критиком этого пакета является Daniel Spiewak, который предлагает свою, улучшенную версию библиотеки для нативной работы с XML: Anti-XML\footnote{http://anti-xml.org/}.

\clearpage
Сходный синтаксис применён и в расширении языка ECMAScript~--- E4X~\cite{E4X} (ECMAScript for XML).

Следующая программа на этом диалекте ECMAScript:

\begin{code}
\begin{lstlisting}[caption={Пример использования XML литеров в языке ECMAScript for XML.}, label=e4x-xml-example]
var sales = <sales vendor="John">
    <item type="peas" price="4" quantity="6"/>
    <item type="carrot" price="3" quantity="10"/>
    <item type="chips" price="5" quantity="3"/>
  </sales>;
console.log(sales.@vendor);
console.log(sales.item[0]);
\end{lstlisting}
выведет на экран \texttt{John} и \texttt{<item type="peas"\ price="4"\ quantity="6"\ />} соответственно.
\end{code}

Стоит отметить, что широкой распространенности расширение языка ECMAScript for XML не получило.
Хотя в некоторых продуктах, таких как Mozilla Firefox\footnote{https://developer.mozilla.org/en/E4X} и
OpenOffice.org\footnote{http://www.ecma-international.org/publications/standards/Ecma-376.htm} существует поддержка этого диалекта.

Отдельно необходимо сказать, что в силу своей природы, ECMAScript не может дать никаких типовых гарантий (как статических, так и динамических) при работе с XML.

\subsubsection{Импортирование типов из схемы XML в программу на языке Kotlin}
Механизм загрузки типов из схем XML в компиляторе Extensible Kotlin схож с механизмом, реализованным в компиляторе языка Gosu.
В программу необходимо подключить XSD файл, описывающий типы:
\begin{code}\begin{lstlisting}[caption={Подключение XML схемы в программу на языке Kotlin.}, label={xsd-type-loading-extension-point}]
fun project() {
  module("org.xsd.shiporder") {
    kotlin extension XsdTypeLoader("shiporder.xsd")
  }
}
\end{lstlisting}\end{code}

\begin{code}
\subsubsection{Примеры использования}
После подключения схемы в программу, загруженные типы станут доступны для использования:

\begin{lstlisting}[caption={Пример использования типов, загруженных из XML схемы в языке Kotlin.}, label=kotlin-simple-xml-example]
val bread = item("Bread", null, 2, 10.0)
val order = shiporder(List<item>(bread))
println(order.toXML())

val fromFile : shiporder = loadFromFile("order.xml")
for (i in fromFile.itemList)
  println(i.title)
\end{lstlisting}\end{code}

\subsubsection{Алгоритм работы загрузчика типов}

Первым делом, еще до компиляции программы, по подключенным схемам XML строятся описания (дескрипторы) классов на языке Kotlin следующего вида:

\begin{code}\begin{lstlisting}[caption={Примеры описаний классов для загруженных типов.}, label=xsd-type-descriptors]
class Item(
  var title : String,
  var quantity : Int,
  var price : Double
)
class Shiporder(
  var orderperson : String,
  var shipto : Shipto,
  val items : List<Item>
)
\end{lstlisting}\end{code}

После подключения XML схемы в пакет компиляции, вся информация о загруженных типах становится доступна в среде разработки. То есть доступна навигация, авто-дополнение, поиск вхождений и тому подобные функции, которые присущи современным средам разработки.

Фазы лексического и синтаксического анализа, построения внутреннего представления программы и первичного анализа типов и разрешения имен полностью унаследованы от базового компилятора языка Kotlin. Но перед генерацией байт-кода присутствуют 2 новых фазы, которых нет в стандартном компиляторе, а именно:

\begin{description}
\item[Трансформация внутреннего представления программы.] При успешном завершении проверок из предыдущего пункта начинается фаза трансформации внутреннего представления программы. Подробное описание алгоритма работы этого шага компиляции можно прочитать в разделе \ref{xml-tranformation-phase}.
\item[Повторный анализ типов и разрешения имен.]
Аналогично пункту, который предшествовал изменению модели, выполняются статические проверки корректности программы с целью выяснения того, что трансформация была выполнена верно. Этот шаг дает дополнительные гарантии корректности модификации программы.
\end{description}

При успешном завершении проверок из последнего пункта генерируется итоговый байт-код.

\subsubsection{Фаза трансформации внутреннего представления программы}\label{xml-tranformation-phase}

Главная суть фазы трансформации~--- заменить все вхождения загруженных типов (в любых позициях: будь то типовой параметр функции, тип возвращаемого значения, тип переменной и т.п.) на один единственный тип класса, который необходим для работы с документом XML.

В данной работе таким универсальным классом стал класс \path{org.jdom.Element} из библиотеки JDOM\footnote{\url{http://www.jdom.org/}}. Библиотека JDOM предоставляет интерфейс для извлечения информации и модификации документов XML. Многолетняя разработка проекта и многочисленное сообщество существенно увеличивают стабильность продукта, а простой программный интерфейс упрощают использование.

В качестве примера можно привести метод \path{getChildren}\footnote{\url{http://www.jdom.org/docs/apidocs/org/jdom/Element.html\#getChildren()}} класса \path{org.jdom.Element}.
Данный метод возвращает изменяемый, а не доступный только для чтения, список своих дочерних элементов. Такой интерфейс позволил легко добиться прозрачной интеграции с коллекциями в языке Kotlin.

Вышеперечисленные факторы и повлияли на выбор пакета JDOM в качестве основного инструмента для манипуляции с объектной моделью документа XML.

Изменение программы затрагивает только те типы, классы которых были синтезированы во время загрузки. В этой секции, если не сказано обратного, подразумевается, что разговор ведется только про такие типы.

Во время модификации программы выполняются следующие преобразования:
\begin{itemize}
\item[---] Все вхождения типов загруженных классов превращаются в тип \path{Element}~--- тип класса, необходимый для операция на деревом XML документа.
\item[---] Все вызовы конструкторов загруженных классов замещаются вызовом конструктора класса \path{Element}.
\item[---] Все обращения на чтение к свойствам загруженных классов (в языке Kotlin нет полей, есть свойства класса) заменяются на обращения к необходимому узлу дерева XML.
\item[---] Аналогично трансформируются операции на запись, такие как присваивание, запись в ячейку массива и т.п.
\end{itemize}

\subsubsection{Результаты}
Реализован механизм загрузки типов из схем XML в компиляторе Extensible Kotlin.

Проведен сравнительный анализ как со сходными механизмами извлечения типов, доступным в языке Gosu, так и со стандартными решениями, которые используются для работы со схемами данных, представленными в виде схем XML (XSD файлов).