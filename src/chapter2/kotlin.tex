\subsection{Язык Kotlin}
Язык Kotlin~--- объектно-ориентированный язык, совместимый с Java и предназначенный для промышленной разработки приложений и компилируемый в переносимый байт-код для виртуальной машины Java. Создание такого языка~--- это ответ на потребность в новом языке, которой был бы с одной стороны полностью совместим с языком Java, а с другой стороны решал бы многочисленные проблемы, которые существуют в языке Java, но не могут быть исправлены по ряду технических причин.

% В последние годы назрела потребность в новом языке, компилируемом в переносимый байт-код для виртуальной машины Java.
% В результате появилось несколько проектов по созданию таких языков, один из которых~--- Kotlin,
% % \footnote{Информация о языке доступна по адресу \url{http://jetbrains.com/kotlin}.}
% статически типизированный объектно-ориентированный язык, совместимый с Java и предназначенный для промышленной разработки приложений.

При создании языка учитывались некоторые важные требования: % TODO: перечитать и удалить лишнее.
\begin{description}
	\item[Совместимость с Java.] Платформа Java~--- это прежде всего экосистема: существует множество продуктов и библиотек, на базе которых строится огромное количество приложений. 
	Поэтому для нового языка очень важна совместимость с уже существующим кодом. Важным моментом является тот факт, что миграция на новый язык может происходить постепенно, таким образом, не только код на Kotlin должен легко вызывать код на Java, но и наоборот.

	\item[Инструментальная поддержка.] Важным условием успешности языка программирования является наличие хорошей инструментальной поддержки. Центральное место среди такимх инстументов занимают интегрированные среды разработки (Integrated Development Environment, IDE). 
	% Десятилетний опыт, накопленный в компании JetBrains, показывает, что определенные свойства языка могут существенно затруднять инструментальную поддержку. 
	При разработке Kotlin, IDE создается одновременно с компилятором. Такой подход позволяет одновременно решить сразу несколько проблем. \td

	\item[Статические гарантии корректности.] Во время компиляции кода на статически типизированном языке происходит множество проверок, призванных гарантировать, что те или иные ошибки не произойдут во время выполнения. Например, компилятор Java гарантирует, что объекты, на которых вызываются те или иные методы, «умеют» их выполнять, то есть что в соответствующих классах эти методы реализованы. К сожалению, кроме этого очень важного свойства, Java почти ничего не гарантирует. Это означает, что успешно скомпилированные программы завершаются с ошибками времени выполнения (вызывают исключительные ситуации). Ярким примером является разыменование нулевой ссылки. Важным требованием к новому языку является усиление статических гарантий. Это позволит обнаруживать больше ошибок на этапе компиляции и, таким образом, сокращать затраты на тестирование.

	\item[Скорость компиляции.] Статические проверки упрощают программирование, но замедляют компиляцию, и здесь необходимо добиться определенного баланса. Опыт создания языков с мощной системой типов (яркими примерами таких языков являются Scala, Haskell) показывает, что такой баланс найти непросто: компиляция зачастую становится неприемлемо долгой.
	Вообще, такая характеристика языка, как время компиляции проекта, может показаться второстепенной, однако в условиях промышленной разработки, когда объемы компилируемого кода очень велики, оказывается, что этот фактор весьма важен~--- ведь пока код компилируется, программист зачастую не может продолжать работу. Известным примером медленной компиляции является язык C++.

	\item[Лаконичность.] Известно \td, что программисты зачастую тратят больше времени на чтение кода, чем на его написание, поэтому важно, чтобы конструкции, доступные в языке программирования, позволяли писать программы кратко и понятно.
	Java считается многословным языком (ceremony language~--- <<церемонный язык>>), и задача Kotlin~--- улучшить ситуацию в этом смысле.
	К сожалению, строгие методы оценивания языков с точки зрения их лаконичности развиты довольно слабо, но есть косвенные критерии; один из них~--- возможность создания библиотек, работа с которыми близка к использованию предметно-ориентированных языков (Domain-Specific Language, DSL). Для создания таких библиотек необходима определенная гибкость синтаксиса в совокупности с конструкциями высших порядков; наиболее распространены функции высших порядков, то есть функции, принимающие другие функции в качестве параметров.

	\item[Доступность для изучения.] Сложные статические проверки, гибкий синтаксис и конструкции высших порядков усложняют язык и затрудняют его изучение, поэтому необходимо в известной степени ограничивать набор поддерживаемых возможностей, чтобы язык был доступен для изучения. При разработке Kotlin учитывался опыт Scala и других современных языков, и слишком сложные концепции в язык не включались.
	
\end{description}