\subsection{Язык Kotlin}
Язык Kotlin~--- объектно-ориентированный язык, совместимый с Java и предназначенный для промышленной разработки приложений и компилируемый в переносимый байт-код для виртуальной машины Java. Создание такого языка~--- это ответ на потребность в новом языке, которой был бы с одной стороны полностью совместим с языком Java, а с другой стороны решал бы многочисленные проблемы, которые существуют в языке Java, но не могут быть исправлены по ряду технических причин.

При создании языка учитывались некоторые важные требования: % TODO: перечитать и удалить лишнее.
\begin{description}
	\item[Совместимость с Java.] Платформа Java~--- это прежде всего экосистема: существует множество продуктов и библиотек, на базе которых строится огромное количество приложений. 
	Поэтому для нового языка очень важна совместимость с уже существующим кодом. Важным моментом является тот факт, что миграция на новый язык может происходить постепенно, таким образом, не только код на Kotlin должен легко вызывать код на Java, но и наоборот.

	\item[Инструментальная поддержка.] Важным условием успешности языка программирования является наличие хорошей инструментальной поддержки. Центральное место среди таких инструментов занимают интегрированные среды разработки (Integrated Development Environment, IDE). 
	При разработке Kotlin, IDE создается одновременно с компилятором. Такой подход позволяет одновременно решить сразу несколько проблем: модифицировать или удалить из языка концепции, которые сложны в поддержке со стороны среды разработки, а также предоставить хорошую инфраструктуру для прикладных программистов с первых дней выпуска продукта.

	\item[Статические гарантии корректности.] Во время компиляции кода на статически типизированном языке происходит множество проверок, призванных гарантировать, что те или иные ошибки не произойдут во время выполнения. Например, компилятор Java гарантирует наличие соответствующего метода у класса объекта, на котором вызывается тот или иной метод. К сожалению, кроме этого очень важного свойства, Java почти ничего не гарантирует. Это означает, что успешно скомпилированные программы завершаются с ошибками времени выполнения. Ярким примером такой исключительной ситуации является разыменование нулевой ссылки. Важным требованием к новому языку является усиление статических гарантий. Это позволит обнаруживать больше ошибок на этапе компиляции и, таким образом, сокращать затраты на тестирование.

	\item[Скорость компиляции.] Статические проверки упрощают программирование, но замедляют компиляцию, и здесь необходимо добиться определенного баланса. Опыт создания языков с мощной системой типов (яркими примерами таких языков являются Scala, Haskell) показывает, что такой баланс найти непросто: компиляция зачастую становится неприемлемо долгой.
	Вообще, такая характеристика языка, как время компиляции проекта, может показаться второстепенной, однако в условиях когда объемы компилируемого кода очень велики, оказывается, что этот фактор весьма важен~--- ведь пока код компилируется, программист зачастую не может продолжать работу. Известным примером медленной компиляции является язык C++.

	\item[Лаконичность.] Известно~\cite{codecomplete}, что программисты зачастую тратят больше времени на чтение кода, чем на его написание, поэтому важно, чтобы конструкции, доступные в языке программирования, позволяли писать программы кратко и понятно.
	Java считается многословным языком (ceremony language~--- <<церемонный язык>>), и задача Kotlin~--- улучшить ситуацию в этом смысле.

	\item[Доступность для изучения.] Сложные статические проверки, гибкий синтаксис и конструкции высших порядков усложняют язык и затрудняют его изучение, поэтому необходимо в известной степени ограничивать набор поддерживаемых возможностей, чтобы язык был доступен для изучения. При разработке Kotlin учитывался опыт Scala и других современных языков, и слишком сложные концепции в язык не включались.
	
\end{description}