\subsection{Язык Kotlin}\label{kotlin-desc}
Язык Kotlin~\cite{breslav2011}~--- объектно-ориентированный язык, предназначенный для промышленной разработки приложений и компилируемый в переносимый байт-код для виртуальной машины Java. Создание такого языка~--- это ответ на потребность в новом языке, которой был бы с одной стороны полностью совместим с языком Java, а с другой стороны решал бы многочисленные проблемы, которые существуют в языке Java, но не могут быть исправлены по ряду технических причин.

При создании языка учитывались некоторые важные требования: % TODO: перечитать и удалить лишнее.
\begin{description}
	\item[Совместимость с Java.] Платформа Java~--- это прежде всего экосистема: существует множество продуктов и библиотек, на базе которых строится огромное количество приложений.
	Поэтому для нового языка очень важна совместимость с уже существующим кодом. Важным моментом является тот факт, что миграция на новый язык может происходить постепенно, таким образом, не только код на Kotlin должен легко вызывать код на Java, но и наоборот.

	\item[Инструментальная поддержка.] Важным условием успешности языка программирования является наличие хорошей инструментальной поддержки. Центральное место среди таких инструментов занимают интегрированные среды разработки (Integrated Development Environment, IDE).
	При разработке Kotlin, IDE создается одновременно с компилятором. Такой подход позволяет одновременно решить сразу несколько проблем: модифицировать или удалить из языка концепции, которые сложны в поддержке со стороны среды разработки, а также предоставить хорошую инфраструктуру для прикладных программистов с первых дней выпуска продукта.

	\item[Статические гарантии корректности.] Во время компиляции кода на статически типизированном языке происходит множество проверок, призванных гарантировать, что те или иные ошибки не произойдут во время выполнения.
	 Ярким примером такой исключительной ситуации является разыменование нулевой ссылки. Важным требованием к новому языку является усиление статических гарантий. Это позволит обнаруживать больше ошибок на этапе компиляции и, таким образом, сокращать затраты на тестирование.

	\item[Скорость компиляции.] Статические проверки упрощают программирование, но замедляют компиляцию, и здесь необходимо добиться определенного баланса. Опыт создания языков с мощной системой типов (яркими примерами таких языков являются Scala~\cite{scala-spec}, Haskell~\cite{haskell98}) показывает, что такой баланс найти непросто: компиляция зачастую становится неприемлемо долгой.
	Вообще, такая характеристика языка, как время компиляции проекта, может показаться второстепенной, однако в условиях когда объемы компилируемого кода очень велики, оказывается, что этот фактор весьма важен~--- ведь пока код компилируется, программист зачастую не может продолжать работу. Известным примером медленной компиляции является язык C++.

	\item[Лаконичность.] Известно~\cite{codecomplete}, что программисты зачастую тратят больше времени на чтение кода, чем на его написание, поэтому важно, чтобы конструкции, доступные в языке программирования, позволяли писать программы кратко и понятно.
	Java считается многословным языком (ceremony language~--- <<церемонный язык>>), и задача Kotlin~--- улучшить ситуацию в этом смысле.

	\item[Доступность для изучения.] Сложные статические проверки, гибкий синтаксис и конструкции высших порядков усложняют язык и затрудняют его изучение, поэтому необходимо в известной степени ограничивать набор поддерживаемых возможностей, чтобы язык был доступен для изучения. При разработке Kotlin учитывался опыт создания других современных языков, и слишком сложные концепции в язык не включались.
\end{description}

\subsubsection{Синтаксические конструкции языка Kotlin, которые необходимы для работы загрузчиков типов}

В этом разделе будут перечислены конструкции языка, синтаксические и семантические соглашения, необходимые для понимания работы механизма загрузки типов в компиляторе Extensible Kotlin.

\begin{description}
\item[Объявления переменных] в языке Kotlin, как и в языке Scala, происходит с помощью ключевых слов \path{val} (для неизменяемых переменных)
и \path{var} (изменяемых). Тип переменной выводится компилятором, но может быть написан и явно:
\begin{code}\begin{lstlisting}
val i = 10
var list = Array<String>(0)
val isOk : Boolean = true
\end{lstlisting}\end{code}
	\item[Классы и первичные конструкторы.] Основными сущностями, с которыми оперирует программист на языке Kotlin, как в других объектно-ориентированных языках являются классы. При объявлении класса список параметров конструктора указывается непосредственно в заголовке:
\begin{code}\begin{lstlisting}
class Point(x : Int, y : Int)
\end{lstlisting}\end{code}
	Создание экземпляра класса происходит с помощью вызова конструктора:
\begin{code}\begin{lstlisting}
val p = Point(10, 20)
\end{lstlisting}\end{code}
	\item[Функции] в языке Kotlin могут существовать не только в внутри классов, но и во вне. При этом они являются такими же полноправными сущностями (first class citizens), как и классы. То есть могут быть переданы как параметр, возвращены из функции, присвоены переменной.
\begin{code}\begin{lstlisting}
fun isEven(i : Int) : Boolean {
    if (i mod 2 == 0)
      return true
    return false
}
fun main(args : Array<String>) {
  val criteria = {(i : Int) -> isEven(i)}
  for (i in 0..10)
    System.out?.println(criteria(i))
}
\end{lstlisting}\end{code}
	\item[Свойства классов.] Параметры конструктора, отмеченные модификатором \path{val}, являются свойствами класса с тем же именем:
\begin{code}\begin{lstlisting}
class Point(val x : Int, val y : Int)

fun main(args : Array<String>) {
  val p = Point(10, 20)
  System.out?.println("x: ${p.x} y: ${p.y}")
}
\end{lstlisting}
	Программа выведет на экран строку: "\texttt{x : 10 y: 20}".
\end{code}
	% \item[Передача функциональных литералов в качестве параметров.]
	% \item[Внешние функции.]
	\item[Работа с нулевыми ссылками.] Типы в Kotlin делятся на содержащие \path{null} и не содержащие \path{null}. Типы, содержащие \path{null}, аннотируются знаком вопроса:
\begin{code}\begin{lstlisting}
fun isPalindrome(s: String?): Boolean {...}
\end{lstlisting}\end{code}

Часто встречаются длинные цепочки вызовов, каждый из которых может вернуть нулевое ссылочное значение~--- \path{null}. В результате мы получаем несколько вложенных условий, проверяющих, что вернул каждый из вызовов в цепочке. Чтобы избежать загромождения кода, в Kotlin поддерживается оператор безопасного вызова, обозначающийся <<\path{?.}>>:
\begin{code}\begin{lstlisting}
a?.getB()?.getC()?.getD()
\end{lstlisting}\end{code}
Если \path{a} не равно \path{null}, выражение \path{a?.getB()} возвращает \path{a.getB()}, иначе~--- \path{null}.

\end{description}