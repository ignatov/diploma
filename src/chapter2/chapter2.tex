\section{Загрузчики типов в языке Extensible Kotlin}
\subsection{Язык Kotlin}
В последние годы назрела потребность в новом языке, компилируемом в переносимый байт-код для виртуальной машины Java.
В результате появилось несколько проектов по созданию таких языков, один из которых~--- Kotlin,
% \footnote{Информация о языке доступна по адресу \url{http://jetbrains.com/kotlin}.}
статически типизированный объектно-ориентированный язык, совместимый с Java и предназначенный для промышленной разработки приложений.

При создании языка учитывались некоторые важные требования: % TODO: перечитать и удалить лишнее.
\begin{description}
	\item[Совместимость с Java.] Платформа Java~--- это прежде всего экосистема: кроме «официальных» продуктов компании Oracle, в нее входит множество проектов с открытым кодом: библиотек и фреймворков разного профиля, на базе которых строится огромное количество приложений. Поэтому для языка, компилируемого для этой платформы, очень важна совместимость с существующим кодом, который написан на Java. При этом необходимо, чтобы существующие проекты могли переходить на новый язык постепенно, то есть не только код на Kotlin должен легко вызывать код на Java, но и наоборот.
	\item[Статические гарантии корректности.] Во время компиляции кода на статически типизированном языке происходит множество проверок, призванных гарантировать, что те или иные ошибки не произойдут во время выполнения. Например, компилятор Java гарантирует, что объекты, на которых вызываются те или иные методы, «умеют» их выполнять, то есть что в соответствующих классах эти методы реализованы. К сожалению, кроме этого очень важного свойства, Java почти ничего не гарантирует. Это означает, что успешно скомпилированные программы завершаются с ошибками времени выполнения (вызывают исключительные ситуации). Ярким примером является разыменование нулевой ссылки, при котором во время выполнения вызывается исключение типа NullPointerException. Важным требованием к новому языку является усиление статических гарантий. Это позволит обнаруживать больше ошибок на этапе компиляции и, таким образом, сокращать затраты на тестирование.
	\item[Скорость компиляции.] Статические проверки упрощают программирование, но замедляют компиляцию, и здесь необходимо добиться определенного баланса. Опыт создания языков с мощной системой типов (наиболее ярким примером является Scala) показывает, что такой баланс найти непросто: компиляция зачастую становится неприемлемо долгой. Вообще, такая характеристика языка, как время компиляции проекта, может показаться второстепенной, однако в условиях промышленной разработки, когда объемы компилируемого кода очень велики, оказывается, что этот фактор весьма важен — ведь пока код компилируется, программист зачастую не может продолжать работу. В частности, быстрая компиляция является одним из важных преимуществ Java по сравнению с C++, и Kotlin должен это преимущество сохранить.
	\item[Лаконичность.] Известно, что программисты зачастую тратят больше времени на чтение кода, чем на его написание, поэтому важно, чтобы конструкции, доступные в языке программирования, позволяли писать программы кратко и понятно. Java считается многословным языком (ceremony language~--- <<церемонный язык>>), и задача Kotlin~--- улучшить ситуацию в этом смысле. К сожалению, строгие методы оценивания языков с точки зрения их лаконичности развиты довольно слабо, но есть косвенные критерии; один из них~--- возможность создания библиотек, работа с которыми близка к использованию предметно-ориентированных языков (Domain-Specific Language, DSL). Для создания таких библиотек необходима определенная гибкость синтаксиса в совокупности с конструкциями высших порядков; наиболее распространены функции высших порядков, то есть функции, принимающие другие функции в качестве параметров.
	\item[Доступность для изучения.] Сложные статические проверки, гибкий синтаксис и конструкции высших порядков усложняют язык и затрудняют его изучение, поэтому необходимо в известной степени ограничивать набор поддерживаемых возможностей, чтобы язык был доступен для изучения. При разработке Kotlin учитывался опыт Scala и других современных языков, и слишком сложные концепции в язык не включались.
	\item[Инструментальная поддержка.] Современные программисты активно используют различные автоматизированные инструменты, центральное место среди которых занимают интегрированные среды разработки (Integrated Development Environment, IDE). Десятилетний опыт, накопленный в компании JetBrains, показывает, что определенные свойства языка могут существенно затруднять инструментальную поддержку. При разработке Kotlin учитывается этот факт и IDE создается одновременно с компилятором.
\end{description}

\subsection{Цель и задачи}

В современных программах зачастую возникает необходимость получать данные из различных источников.
По мере усложнения приложения могут потребоваться статические проверки при доступе к этим данным.

Другими словами, возникает необходимость представления информации из внешних источников в виде типов, свойств и методов языка программирования.
Написание этих типов руками отнимает большое количество времени, сил, также возникают сложности в поддержки сопутствующего кода.
Стандартное решение в данной ситуации~--- использование кодогенераторов, которые добавляют необходимые файлы в проект.

Загрузчики типов являются альтернативой стандартным решениям и позволяют расширять систему типов языков без написания (или генерации) дополнительного кода.
Что положительным образом влияет как на скорость разработки программного обеспечения, так и на стоимость поддержки проекта.

В данной работе предлагается расширение языка Kotlin~--- Extensible Kotlin, которое позволяет создавать такие загрузчики.

\subsection{Архитектура компилятора Extensible Kotlin}
Компилятор Extensible Kotlin построен на основе компилятора Kotlin.

Рассмотрим стандартные шаги компиляции:
\begin{itemize}
\item[---] Лексический разбор;
\item[---] Грамматический разбор;
\item[---] Построение внутреннего представления программы;
\item[---] Статический анализ; \td
\item[---] Генерация байт-кода (при успешном завершении проверок из предыдущего пункта);
\end{itemize}

Сущетсвенным отличием компилятора Extensible Kotlin от компилятора Kotlin является дополнительная фаза компиляции трансформации внутреннего представления программы  и последующие статические проверки. То есть шаги компиляции программы с помощью компилятора Extensible Kotlin выглядят следующим образом:
\begin{itemize}
\item[---] Лексический разбор;
\item[---] Грамматический разбор;
\item[---] Построение внутреннего предствления программы;
\item[---] Статический анализ; \td
\item[---] Трансформация внутренного предствления программы (при успешном завершении проверок из предыдущего пункта);
\item[---] Статический анализ вновь полученных структур; \td
\item[---] Генерация байт-кода (при успешном завершении проверок из предыдущего пункта);
\end{itemize}

\subsubsection{Расширяемые компиляторы в других языках} % TODO: нужно подробнее
Ряд современных языков программирования поддерживает компиляцию в несколько отдельных фаз.
В основном, такие языки являются расширениями уже существующих языков: MetaOCaml, MetaMl, 'C, но есть и самостоятельные, как например, Scala. \td

% 'C     - http://pdos.csail.mit.edu/~engler/pldi96-abstract.html
% MetaML - esop99.pdf, 				An Idealized MetaML: Simpler, and More Expressive
% Scala  - compiler-phases-sid.pdf, Scala Compiler Phase and Plug-In Initialization for Scala 2.8

Компиляторы таких языков позволяют управлять ходом компиляции программы с помощью специальных пользовательских расширений (Compiler Plugins \td). Стоит отметить, что на сегодняшний день в компиляторе языка Scala существует 21 фаза компиляции и поведение каждой из них можно изменить.

% По аналогичному принципу построена работа компилятора языка Extensible Kotlin.
\subsection{Загрузчик типов из схем XML}\label{xml-loader}

\todo{Пустая секция, нужно дописать мини-введение.}

\subsubsection{Пример входных данных}

Представим, что у нас есть следующая схема, описывающая XML файлы, которые мы хотим считывать в нашей программе:
\begin{code}
\lstinputlisting[caption={Пример XML cхемы.}, label=xsd-example-input]{listings/shiporder.xsd}
\end{code}

После изучения схемы видно, что у элемента \path{shiporder} есть список дочерних элементов \path{item}, у каждого из которых есть три обязательных поля: строковое \path{title}, положительное целое \path{quantity}, десятичное \path{price} (описывающее стоимость товара) и одно необязательное строковое~--- \path{note}.

Подобные рассмотрения приводят к мысли, что схемы данных XML могут один к одному транслироваться в классы языка программирования и обратно.

Следовательно, имеет смысл написание специального расширения компилятора языка, которое позволит автоматически загружать типы данных из файлов со схемами и оперировать со статически типизированной информацией, которая согласована с этими схемами.

Пример файла, удовлетворяющей схеме, приведенной выше:
\begin{code}
\lstinputlisting[caption={Пример XML файла, удовлетворяющего схеме из примера~\ref{xsd-example-input}.}, label=xml-for-xsd-example]{listings/order.xml}
\end{code}

\subsubsection{Примеры языков со встроенной поддержкой XML}
Некоторые современные языки программирования обладают встроенной поддержкой XML.

Например, в уже упомянутом языке Scala~\cite{scala-spec}, литералы XML являются частью синтаксиса:

\begin{code}\begin{lstlisting}[caption={Пример использования литералов XML в языке Scala.}, label=scala-xml-example]
val document =
  <root>
    <child>
      <grandchild an_attribute="value1" />
      <grandchild an_attribute="value2" />
    </child>
  </root>
\end{lstlisting}\end{code}
Во внутреннем представлении компилятора каждый XML--литерал является экземпляром класса \path{scala.xml.Elem}\footnote{http://www.scala-lang.org/api/current/scala/xml/Elem.html}:

\begin{code}\begin{lstlisting}[caption={Внутреннее представление XML литералов из примера~\ref{scala-xml-example}.}, label={scala-xml-example-internal}]
val document =
scala.xml.Elem(null, "root", scala.xml.Null, scala.xml.TopScope,
  scala.xml.Elem(null, "child", scala.xml.Null, scala.xml.TopScope,
    scala.xml.Elem(null, "grandchild",
      new scala.xml.UnprefixedAttribute(
        "an_attribute","value1", scala.xml.Null),
      scala.xml.TopScope,
      scala.xml.Text("content1")
    ),
    scala.xml.Elem(null, "grandchild",
      new scala.xml.UnprefixedAttribute(
        "an_attribute","value2", scala.xml.Null),
      scala.xml.TopScope,
      scala.xml.Text("content2")
    )
  )
)
\end{lstlisting}\end{code}

В языке Scala для работы с XML предлагается достаточно мощный интерфейс. Именно он и используется в многочисленных каркасах для создания веб-приложений (web application frameworks), написанных на Scala.

С точки зрения подобного класса программ (библиотек для создания веб--приложений), нативная поддержка XML
литералов в языке является несомненным плюсом, нежели минусом, позволяя прозрачно работать с пользовательскими шаблонами.

Также стоит отметить критику пакета \path{scala.xml} о недостаточной производительности, несогласованности некоторых методов и т.п.
Главным критиком этого пакета является Daniel Spiewak, который предлагает свою, улучшенную версию библиотеки для нативной работы с XML: Anti-XML\footnote{http://anti-xml.org/}.

Сходный синтаксис применён и в расширении языка ECMAScript~--- E4X~\cite{E4X} (ECMAScript for XML).

\begin{code}
Следующая программа на этом диалекте ECMAScript:

\begin{lstlisting}[caption={Пример использования XML литеров в языке ECMAScript for XML.}, label=e4x-xml-example]
var sales = <sales vendor="John">
    <item type="peas" price="4" quantity="6"/>
    <item type="carrot" price="3" quantity="10"/>
    <item type="chips" price="5" quantity="3"/>
  </sales>;
console.log(sales.@vendor);
console.log(sales.item[0]);
\end{lstlisting}
выведет на экран \texttt{John} и \texttt{<item type="peas"\ price="4"\ quantity="6"\ />} соответственно.
\end{code}

Стоит отметить, что широкой распространенности расширение языка ECMAScript for XML не получило.
Хотя в некоторых продуктах, таких как Mozilla Firefox\footnote{https://developer.mozilla.org/en/E4X} и
OpenOffice.org\footnote{http://www.ecma-international.org/publications/standards/Ecma-376.htm} существует поддержка этого диалекта.

Отдельно необходимо сказать, что в силу своей природы, ECMAScript не может дать никаких типовых гарантий (как статических, так и динамических) при работе с XML.

\subsubsection{Импортирование типов из схемы XML в программу на языке Kotlin}
Механизм загрузки типов из схем XML в компиляторе Extensible Kotlin схож с механизмом, реализованным в компиляторе языка Gosu.
В программу необходимо подключить XSD файл, описывающий типы:
\begin{code}\begin{lstlisting}[caption={Подключение XML схемы в программу на языке Kotlin.}, label={xsd-type-loading-extension-point}]
#TODO
\end{lstlisting}\end{code}

\subsubsection{Примеры использования}
\begin{code}
После подключения схемы в программу, загруженные типы станут доступны для использования:

\begin{lstlisting}[caption={Пример использования типов, загруженных из XML схемы в языке Kotlin.}, label=kotlin-simple-xml-example]
val bread = item("Bread", null, 2, 10.0)
val order = shiporder(List<item>(bread))
println(order.toXML())

val fromFile : shiporder = loadFromFile("order.xml")
for (i in fromFile.itemList)
  println(i.title)
\end{lstlisting}\end{code}

\subsubsection{Алгоритм работы загрузчика типов}

Первым делом, еще до компиляции программы, по подключенным схемам XML строятся описания (дескрипторы) классов на языке Kotlin следующего вида:

\begin{code}\begin{lstlisting}[caption={Примеры описаний классов для загруженных типов.}, label=xsd-type-descriptors]
class Item(
  var title : String,
  var quantity : Int,
  var price : Double
)
class Shiporder(
  var orderperson : String,
  var shipto : Shipto,
  val items : List<Item>
)
\end{lstlisting}\end{code}

После подключения XML схемы в пакет компиляции, вся информация о загруженных типах становится доступна в среде разработки. То есть доступна навигация, авто--дополнение, поиск вхождений и тому подобные функции, которые присущи современным средам разработки.

Фазы лексического и синтаксического анализа, построения внутреннего представления программы и первичного анализа типов и разрешения имен полностью унаследованы от базового компилятора языка Kotlin. Но перед генерацией байт-кода присутствуют 2 новых фазы, которых нет в стандартном компиляторе, а именно:

\begin{description}
\item[Трансформация внутреннего представления программы.] При успешном завершении проверок из предыдущего пункта начинается фаза трансформации внутреннего представления программы. Подробное описание алгоритма работы этого шага компиляции можно прочитать в разделе \ref{xml-tranformation-phase}.
\item[Повторный анализ типов и разрешения имен вновь полученных структур.]
Аналогично пункту, который предшествовал изменению модели, выполняются статические проверки корректности программы с целью выяснения того, что трансформация была завершена корректно. Этот шаг дает дополнительные гарантии корректности модификации программы.
\end{description}

При успешном завершении проверок из последнего пункта генерируется итоговый байт-код.

\subsubsection{Фаза трансформации внутреннего представления программы}\label{xml-tranformation-phase}
\todo{Убрать повторы: тип...}

Главная суть фазы трансформации~--- заменить все вхождения загруженных типов (в любых позициях: будь то типовой параметр функции, тип возвращаемого значения, тип переменной и т.п.) на один единственный, который необходим для работы с документом XML.

В данной работе таким универсальным классом стал класс \path{org.jdom.Element} из библиотеки JDOM\footnote{\url{http://www.jdom.org/}}. Библиотека JDOM предоставляет интерфейс для извлечения информации и модификации документов XML. Многолетняя разработка проекта и многочисленное сообщество существенно увеличивают стабильность продукта, а простой, неперегруженный интерфейс упрощают использование.

В качестве примера можно привести метод \path{getChildren}\footnote{\url{http://www.jdom.org/docs/apidocs/org/jdom/Element.html\#getChildren()}} класса \path{org.jdom.Element}.
Данный метод возвращает изменяемый, а не доступный только для чтения, список своих дочерних элементов. Такой интерфейс позволил легко добиться прозрачной интеграции с коллекциями в языке Kotlin.

Вышеперечисленные факторы и повлияли на выбор пакета JDOM в качестве основного инструмента для манипуляции с объектной моделью документа XML.

Изменение программы затрагивает только те типы, классы, которых были сгенерированы во время загрузки. В этой секции, если не сказано обратного, подразумевается, что разговор ведется только про такие типы.

Во время модификации программы выполняются следующие преобразования:
\begin{itemize}
\item[---] Все вхождения типов загруженных классов превращаются в тип \path{Element}~--- тип класса, необходимый для операция на деревом XML документа.
\item[---] Все вызовы конструкторов загруженных классов замещаются вызовом конструктора класса \path{Element}.
\item[---] Все обращения на чтение к свойствам загруженных классов (в языке Kotlin нет полей, есть свойства класса) заменяются на обращения к необходимому узлу дерева XML.
\item[---] Аналогично трансформируются операции на запись в свойства загруженных классов.
\end{itemize}

\subsubsection{Результаты}
Реализован механизм загрузки типов из схем XML в компиляторе языка Extensible Kotlin.

Проведен сравнительный анализ как со сходными механизмами извлечения типов, доступными в языках F\# и Gosu, так и со стандартными решениями, которые используются для работы со схемами данных, представленными в виде схем XML (XSD файлов).
\subsection{Загрузчик типов из заголовочных файлов на языке С}

Иногда перед прикладным программистом ставится задача перенесения существующего кода на новую программную платформу.
В ходе таких работ может возникнуть необходимость в вызове кода из библиотек, доставшихся в наследство (legacy code).

Для таких случаев язык Extensible Kotlin предоставляет возможность загрузки структур данных и функций из заголовочных файлов на языке C.
При загрузке, структуры данных преобразуются в классы, а функции~--- в функции.

\subsubsection{Постановка задачи}
\subsubsection{Пример входных данных}
Рассмотрим следующий пример. Представим, что у нас есть библиотека на языке C, а также её заголовочный файл со следующим содержанием:
\lstinputlisting[language=C]{listings/point.h}

\subsubsection{Импортирование заголовочного файла на языке C программу на языке Extensible Kotlin}
Аналогично подключению загрузчика типов для схем XML, подключение происходит явно. \td

\subsubsection{Примеры использования}
После подключения загрузчика и импорта загруженных классов и функций можно приступать к вызову кода на языке C из программы на языке Kotlin:
\begin{lstlisting}
import org.libs.point.Point, move, copy_with_offset

fun main(args: Array<String>) {
  val zero: Point = Point(0, 0)
  move(zero, 100, 100)
  val result = copy_with_offset(zero, 100, 100)
  print("x: " + result.x + ", y: " + result.y)
}
\end{lstlisting}
% Стоит отметить, что в данной реализации загрузчика классы-обёртки для взаимодействия с кодом на языке C создаются, упаковываются в конечную сборку и вызываются во время исполнения программы.
\subsubsection{Алгоритм работы загрузчика функций и типов}
\subsubsection{Результаты}