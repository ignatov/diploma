\subsection{Архитектура компилятора Extensible Kotlin}
Компилятор Extensible Kotlin построен на основе компилятора Kotlin.

Рассмотрим стандартные шаги компиляции:
\begin{itemize}
\item[---] Лексический разбор;
\item[---] Грамматический разбор;
\item[---] Построение внутреннего представления программы;
\item[---] Статический анализ; \td
\item[---] Генерация байт-кода (при успешном завершении проверок из предыдущего пункта);
\end{itemize}

Сущетсвенным отличием компилятора Extensible Kotlin от компилятора Kotlin является дополнительная фаза компиляции трансформации внутреннего представления программы  и последующие статические проверки. То есть шаги компиляции программы с помощью компилятора Extensible Kotlin выглядят следующим образом:
\begin{itemize}
\item[---] Лексический разбор;
\item[---] Грамматический разбор;
\item[---] Построение внутреннего предствления программы;
\item[---] Статический анализ; \td
\item[---] Трансформация внутренного предствления программы (при успешном завершении проверок из предыдущего пункта);
\item[---] Статический анализ вновь полученных структур; \td
\item[---] Генерация байт-кода (при успешном завершении проверок из предыдущего пункта);
\end{itemize}

\subsubsection{Расширяемые компиляторы в других языках} % TODO: нужно подробнее
Ряд современных языков программирования поддерживает компиляцию в несколько отдельных фаз.
В основном, такие языки являются расширениями уже существующих языков: MetaOCaml \td, MetaMl~\cite{metaml}, 'C~\cite{extendible-c}, но есть и самостоятельные, как например, Scala~\cite{scala-spec}. 

Компиляторы таких языков позволяют управлять ходом компиляции программы с помощью специальных пользовательских расширений (Compiler Plugins~\cite{scala-compiler-plugin}). Стоит отметить, что на сегодняшний день в компиляторе языка Scala существует 21 фаза компиляции и поведение каждой из них можно изменить.

% По аналогичному принципу построена работа компилятора языка Extensible Kotlin.