\subsection{Цель и задачи}

В современных программах зачастую возникает необходимость получать данные из различных источников.
По мере усложнения приложения могут потребоваться статические проверки при доступе к этим данным.

Другими словами, возникает необходимость представления информации из внешних источников в виде типов, свойств и методов языка программирования.
Написание этих типов вручную отнимает большое количество времени, сил, также возникают сложности поддержки сопутствующего кода.
Стандартное решение в данной ситуации~--- использование кодогенераторов, которые добавляют необходимые файлы в проект.

Целью данной работы является разработка универсального механизма взаимодействия с внешними по отношению к языку интерфейсами~--- загрузчика типов. Данный механизм положительным образом влияет как на скорость разработки программного обеспечения, так и на стоимость поддержки проекта.

Для достижения намеченной цели были поставлены и решены следующие задачи:
\begin{itemize}
\item[---] Реализовать загрузчик типов из декларативно-описанных XML схем;
\item[---] Реализовать загрузчик типов из заголовочных файлов на языке C;
\item[---] Сравнить созданные механизмы с существующими программными продуктами, призванными решать подобные задачи;
\end{itemize}