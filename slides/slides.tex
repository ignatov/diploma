\documentclass[14pt]{beamer}
\usepackage[english,russian]{babel}
\usepackage[utf8]{inputenc}
\usepackage{listings}

\setbeamersize{text margin left=6mm, text margin right=6mm}

\linespread{1.2}

% Стиль презентации
\usetheme{Amsterdam}

%gets rid of bottom navigation bars
\setbeamertemplate{footline}[page number]{}

%gets rid of navigation symbols
\setbeamertemplate{navigation symbols}{}

\beamertemplatenavigationsymbolsempty

\begin{document}
\title{Синтез объектно-ориентированных интерфейсов из декларативных описаний форматов данных и библиотек}
\author{Игнатов С.С. группа 6057/1}
% \institute{Санкт-Петербургский государственный политехнический университет \\
% Физико-механический факультет \\
% Кафедра прикладной математики}
\date{Санкт-Петербург, 2012}

\frame{\titlepage}

% \frame{\frametitle{Содержание}\tableofcontents[currentsection]}

\begin{frame}\frametitle{Источники данных}
    \begin{itemize}
        \item[---] XSD, WSDL
        \item[---] Protobuf
        \item[---] Схемы баз данных
        \item[---] Заголовочные файлы на языке C
    \end{itemize}
    \small{Необходимо наличие декларативного описания (схемы).}
\end{frame}

\begin{frame}\frametitle{Постановка задачи}
    Разработать механизм работы с внешними источниками данных.

    Важные свойства:
    \begin{itemize}
        \item[---] Статическая типизация
        \item[---] Синхронизация с интерфейсом
        \item[---] Удобство использования
        \item[---] Поддержка в среде разработки
    \end{itemize}
    Создать реализации для работы с XML и языком C.

\end{frame}

\begin{frame}\frametitle{Обзор существующих решений} % TODO
    \begin{itemize}
        \item[---] Поставщики типов (type providers) в языке F\#
        \item[---] Загрузчики типов (type loaders) в языке Gosu
        \item[---] Схожие механизмы:
            \begin{itemize}
                \item[---] Pluggable Types
                \item[---] Представление данных в языке EL1
            \end{itemize}
    \end{itemize}
\end{frame}

\begin{frame}\frametitle{Язык программирования Kotlin}
    \begin{itemize}
        \item[---] Объектно-ориентированный
        \item[---] Статически типизированный
        \item[---] Компилируется в байт-код для JVM
        \item[---] Общего назначения
    \end{itemize}
\end{frame}

\begin{frame}\frametitle{Архитектура компилятора}
\linespread{0.9}
\begin{small}
    Расширяемые компиляторы: MetaML, MetaOcaml, Haskell, Scala, Gnu Compiler Collection.

    Фазы компиляции:
    \begin{itemize}
        \item[---] Лексический анализ
        \item[---] Синтаксический анализ
        \item[---] Построение внутреннего представления программы
        \item[---] Разрешение имен и анализ типов
        \item[---] \textbf{Трансформация внутреннего представления программы}
        \item[---] \textbf{Разрешение имен и анализ типов вновь полученных структур}
        \item[---] Генерация байт-кода
    \end{itemize}
\end{small}
\linespread{1.2}
\end{frame}

\begin{frame}\frametitle{Пример использования XML}
\end{frame}

\begin{frame}\frametitle{Пример использования C}
\end{frame}

\begin{frame}\frametitle{Основные результаты}
    Качественные улучшения:
    \begin{itemize}
        \item[---] Согласованность с внешними интерфейсами
        \item[---] Упрощение рабочего процесса программиста
        \item[---] Отсутствие сгенерированных артефактов в системе контроля версий
    \end{itemize}
\end{frame}

\begin{frame}\frametitle{Количественные улучшения}
    \begin{itemize}
        \item[---] Лаконичный код:
            \begin{itemize}
                \item[---] сопоставим с Gosu
                \item[---] в 2--3 раза меньше, чем Java
            \end{itemize}
        \item[---] Уменьшение размера бинарной сборки за счет удаления типовой информации
    \end{itemize}
\end{frame}

\begin{frame}\frametitle{Выводы}
    \begin{itemize}
        \item[---] Реализовано 2 механизма загрузки типов
        \item[---] Достигнуты качественные улучшения
        \item[---] Отсутствие сгенерированных артефактов в системе контроля версий
    \end{itemize}
\end{frame}

\end{document}

% listing example
% \frame[containsverbatim]{
% \frametitle{Source code}

% \begin{lstlisting}[language=C]
% int main() {
%     printf("Hello World!");
%     return 0;
% }
% \end{lstlisting}
% }
